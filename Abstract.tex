%% Las secciones del "prefacio" inician con el comando \prefacesection{T'itulo}
%% Este tipo de secciones *no* van numeradas, pero s'i aparecen en el 'indice.
%%
%% Si quieres agregar una secci'on que no vaya n'umerada y que *tampoco*
%% aparesca en el 'indice, usa entonces el comando \chapter*{T'itulo}
%%
%% Recuerda que aqu'i ya puedes escribir acentos como: 'a, 'e, 'i, etc.
%% La letra n con tilde es: 'n.

\chapter*{Abstract}

%\thispagestyle{empty}

Crop improvement is the result of genetic development which requires several years and large investments. In advanced stages of breeding programs, multi-environment trials (MET), which consist of evaluating different cultivars in multiple environments, are essential tools to increase crop productivity. Since varieties remain on market for decades, their choice is essential to avoid economic losses due to bad seasons and to ensure a constant supply. Consequently, an adequate analysis of MET data is essential to guarantee the success of a breeding program. Currently, R is one of the most widely used programming language for data analysis due to its distribution as free software and the wide variety of tools it offers. However, breeders who are unfamiliar with programming tend to use other types of programs that respond to menu prompts instead of writing lines of code, despite the financial costs of their licenses. Whereas, those who have an affinity with the use of code for data analysis face difficulties in identifying the right tools from the large number of instruments available. Therefore, in this work two tools are develop for MET data analysis. On one hand, a new R package that includes new methodology not available in the software and at the same time brings together all those most useful created to facilitate the users task. On the other hand, a graphical user interface was created using a Shiny web application that allows the main analyzes implemented in the package to be carried out without the need for programming and is published on internet for free access. 

\textbf{Keywords}: multi-environment trials, programming, statistical analysis, user interfaz, R languaje.

%% Por si alguien tiene curiosidad, este "simp'atico" agradecimiento est'a
%% tomado de la "Tesis de Lydia Chalmers" basada en el universo del programa
%% de televisi'on Buffy, la Cazadora de Vampiros.
%% http://www.buffy-cazavampiros.com/Spiketesis/tesis.inicio.htm
