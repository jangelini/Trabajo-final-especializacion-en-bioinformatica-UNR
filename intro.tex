%% Los cap'itulos inician con \chapter{T'itulo}, estos aparecen numerados y
%% se incluyen en el 'indice general.
%%
%% Recuerda que aqu'i ya puedes escribir acentos como: 'a, 'e, 'i, etc.
%% La letra n con tilde es: 'n.


\chapter{Introducción}

A lo largo de la historia de la agricultura, el hombre ha desarrollado el mejoramiento vegetal en forma sistemática y lo ha convertido en un instrumento esencial para incrementar la producción agrícola en términos de cantidad, calidad y diversidad.  

El fitomejoramiento, en un sentido amplio, busca alterar la frecuencia alélica de los genes para obtener cultivares  genéticamente superiores, adaptados a condiciones específicas, con mayor rendimiento y mejor calidad que las variedades nativas o criollas \citep{Allard67}. En otras palabras, su objetivo es desarrollar genotipos cuya superioridad genética esté de acuerdo con las condiciones agroclimáticas donde se producen, necesidades y recursos de todos aquellos que elaboran, transforman y consumen productos vegetales. 

Las variedades mejoradas son el resultado del trabajo llevado a cabo en los programas de fitomejoramiento, los cuales se extienden a lo largo de varios años y requieren cuantiosas inversiones. Generalmente, en etapas tempranas de estos programas existe un gran número de genotipos experimentales con pocos antecedentes de evaluación; mientras que en etapas posteriores  se evalúa un número menor con más repeticiones y en más ambientes/años. Estos ensayos multiambientales (EMA) son herramientas fundamentales para evaluar la productividad para así asegurar la rentabilidad de los cultivos.

Como consecuencia de que los EMA se llevan a cabo en múltiples ambientes/años, la aparición de la interacción genotipo $\times$ ambiente (IGA) es inevitable debido a las variaciones en las condiciones climáticas y de suelo. La IGA es considerada por los fitomejoradores como el principal factor factor limitante de  respuesta a la selección y, en general, de la eficiencia de los programas de mejoramiento, por provocar respuestas altamente variables en los diferentes ambientes \citep{Crossaetal1990, CruzMedina1992}; \textbf{Kang y Magari, 1996)}. \citet{GauchZobel1997} explicaron que si no hubiera interacción, una sola variedad / híbridos rendirían al máximo en todo el mundo, además los materiales podrían evaluarse en un solo lugar y proporcionarían resultados universales.


\textbf{Peto (1982)} ha distinguido las interacciones cuantitativas, conocida también como sin cambio de rango o \emph{no crossover} (NCOI), de las cualitativas, denominada también con
cambio de rango o \emph{crossover} (COI) (Cornelius et al., 1996). Cuando dos genotipos X e Y tienen una respuesta diferencial en dos ambientes, y hay cambios en el orden de los genotipos se dice que la IGA es del tipo COI (Figura \ref{fig:fig11}(A)), es NCOI cuando su ordenación permanece sin cambios (Figura  \ref{fig:fig11}(B)), y. finalmente, es inexistente cuando los genotipos responden de manera similar en ambos ambientes (Figura \ref{fig:fig11}(C)). 


\begin{figure}[h]
\begin{center}
\includegraphics[width=14cm]{./Graficos/interac}
\end{center}
\caption{Representación gráfica de tipos de IGA: (A)IGA crossover, (B) IGA no crossover y (C) no IGA}
\label{fig:fig11}
\end{figure}


Entre las implicancias negativas de la IGA en los programas de mejoramiento se encuentra el impacto negativo sobre la heredabilidad, cuanto menor sea la heredabilidad de un caracter, mayor será la dificultad para mejorarlo. Como consecuencia, la información sobre la estructura y la naturaleza de la IGA es particularmente útil para determinar si se deben desarrollar cultivares con adaptación amplia o específica (\textbf{Bridges, 1989}). La decisión sobre qué tipo de estrategia seguir, involucra considerar y analizar  conceptos como regiones ecológicas, ecotipos y mega-ambientes (\textbf{Kang et al., 2004}).


La vigencia comercial de las variedades puede extenderse durante varias décadas, por lo que su elección es crítica para que el productor evite pérdidas económicas por malas campañas y el suministro al mercado sea constante. Consecuentemente, un análisis adecuado de la información de los EMA es indispensable para que el programa de mejoramiento genético de los cultivos sea eficaz. El rendimiento medio en los ambientes es un indicador suficiente del rendimiento genotípico sólo en ausencia de IGA \citep{YanKang2003}. Sin embargo, la aparición de IGA es inevitable y no basta con la comparación de las medias de los genotipos, sino que se debe recurrir a una metodología estadística más aporopiada. La metodología estadística más difundida para analizar los datos provenientes de EMA se basa en modificaciones de los modelos de regresión, análisis de variancia (ANOVA) y técnicas de análisis multivariado. 

Particularmente, para el estudio de la IGA y los análisis que de ella se derivan, dos modelos multiplicativos han aumentado su popularidad entre los fitomejoradores, especialmente como una herramienta de análisis gráfico: el modelo de los efectos principales aditivos e interacción multiplicativa (\emph{Additive Main effects and Multiplicative Interaction}, AMMI) \citep{Kempton1984,Gauch1988} , y el de regresión por sitio (\emph{Site Regression model}, SREG) (\textbf{Cornelius et al., 1996}; \citep{GauchZobel1997}.  Estos modelos combinan un ANOVA con la descomposición de valores singulares (DVS) o un análisis de componentes principales (ACP) sobre la matriz residual de ANOVA. En SREG, el ANOVA se realiza sobre el efecto ambiental (A),  mientras que en AMMI el ANOVA se realiza sobre los efectos principales de genotipos (G) y A. A partir de estos modelos se pueden visualizar los patrones puramente atribuibles a los efectos la IGA, mediante el biplot derivado del modelo AMMI y de aquel obtenido del SREG explorar simultáneamente patrones de variación conjunta de G e IGA (\textbf{Yan y Hunt (2002)}). 

Una limitación importante de la mayoría de las propuestas del análisis de EMA es que requieren que el conjunto de datos esté completo. Aunque los EMA están diseñados para que la totalidad de los genotipos se evalúen en todos los ambientes,  las tablas de datos genotipo $\times$ ambiente completas son poco frecuentes (no todos los genotipos se encuentran en todos los ambientes). Esto ocurre, por ejemplo, debido a errores de medición o pérdidas de plantas por presencia de animales, inundaciones o problemas durante la cosecha, además de la dinámica propia de la evaluaciones en las que se incorporan y se descartan genotipos debido a su pobre desempeño (Hill y Rosemberg, 1985). En estos casos, antes de llevar a cabo el análisis de interés resulta necesario imputar los datos faltantes con metodología apropiada(\textbf{CITA}). 


En este contexto, el análisis de datos provenientes de EMA requiere metodología estadística cuyas rutinas informáticas no se encuentran disponibles en programas comerciales debido a su reciente desarrollo o bien, se deben utilizar varios de ellos para lograr un único objetivo. Esto último genera el inconveniente de tener que disponer de todos los programas necesarios para los distintos análisis, atender los requerimientos de formatos de datos usados por cada uno de ellos, y comprender los diversos tipos de salidas en las que se presentan los resultados obtenidos. Además, los costos de las licencias algunos programas pueden resultar muy elevados. 

El software R, desarrollado por \emph{The R Foundation for Statistical Computing}, se trata de un proyecto colaborativo de uso libre y distribuido bajo los términos de la \emph{General Public Licence} (GNU). Este software surge como resultado de la implementación del lenguaje S, uno de los más utilizados en investigación por la comunidad estadística. A diferencia de los programas utilizados frecuentemente para el análisis de datos, al ser R un lenguaje de programación, no dispone de una interfaz gráfica que facilite su uso mediante menús, generando cierta dificultad para aquéllos que no se encuentran familiarizados con la programación. Sin embargo, por ser un software libre, permite a los usuarios definir sus propias funciones dando lugar a mayores posibilidades respecto del manejo y análisis de los datos disponibles. Si bien la versión básica del programa dista mucho de ser amigable, RStudio, un entorno de desarrollo integrado (IDE) gratuito y de código abierto para R, permite una interacción más fluida con el programa, actuando como una interfaz con el usuario.  Al formar parte de un proyecto colaborativo, R promueve el hecho de que los usuarios compartan con la comunidad las funciones creadas por ellos, por lo que está en continuo desarrollo y actualización. A menudo, no resulta sencillo reutilizar una función creada por algún usuario, por ello, se ha introducido la posibilidad de crear paquetes (\emph{package}). Éstos son una colección de objetos desarrollados y organizados siguiendo un protocolo fijo, lo cual garantiza un soporte mínimo para el usuario así como la ausencia de errores (de sintaxis) en la programación.

Actualmente, R cuenta con 14 paquetes básicos y 29 recomendados para su funcionamiento que se instalan automaticamente en él, tal como \emph{base} o \emph{stats}. Entre los paquetes que extienden las funciones básicas de R se encuentran, \emph{plyr}, \emph{lubridate}, \emph{reshape2} y \emph{stringr} para la manipulación de los datos; \emph{ggplot2} y \emph{rgl} para la visualización; \emph{knitr} y \emph{xtable} para la presentación de resultados; entre otros. La lista completa de los paquetes oficiales puede consultarse en CRAN\footnote{CRAN (Comprehensive R Archve Network) es el repositorio oficial de paquetes de R, el lugar donde se publican las nuevas versiones del programa, etc. Contiene la lista completa de paquetes oficiales. \url{https://cran.r-project.org/web/packages/available_packages_by_name.html}}, que contaba con más de 14.000 paquetes disponibles hasta junio de 2019. Esta gran variedad de paquetes es una de las razones de la gran difusión de R, ya que cada usuario puede tratar su problematica utilizando un paquete desarrollado por otro usuario. Además de los paquetes oficiales, existen otros que pueden instalarse desde repositorios como Github, Bioconductor, rOpenSci, entre otros. Sin embargo, no es sencillo encontrar un paquete que incluya todas las funciones necesarias para un cierto análisis, sino que debe recurrirse a varios de ellos. 

Con frecuencia, los mejoradores usan programas de análisis de datos que cuentan con interfaz gráfica de usuario, evitando el manejo de un lenguaje de programación. Hoy en día las aplicaciones web son muy populares debido a la facilidad de su uso, ya que no requiere de una instalación en el ordenador del usuario, simplemente se accede a través de un navegador siendo posible utilizarlas desde cualquier dispositivo con conexión a internet.

Crear aplicaciones web puede resultar difícil para la mayoría de los usuarios de R debido a que se necesita un conocimiento profundo de otros lenguajes de programación como HTML, CSS y JavaScript; y además el desarrollo de aplicaciones interactivas requiere un análisis cuidadoso de los flujos de trabajo para asegurarse de que cuando una entrada cambie, solo se actualicen las salidas relacionadas. Sin embargo,  Winston C., et al. (2012) crearon el paquete \emph{shiny} de R que facilita el desarrollo de aplicaciones web, acercando la potencia del software a todo tipo de usuarios sin tener la necesidad de conocer el lenguaje de programación. Este paquete le facilita al programador de R el proceso de creación de 
aplicaciones web al proporcionar un conjunto de funciones de interfaz de usuario que generan el HTML, CSS y JavaScript sin necesidad de conocer los detalles de dichos lenguajes. 



El objetivo del presente trabajo es: (i) desarrollar un paquete de R para el análisis de datos provenientes de EMA que incluya metodologías existentes, modificadas para favorecer su uso, y otras recientemente publicadas y no disponibles en R y (ii) desarrollar una aplicación web Shiny que sirva como interfaz gráfica de usuario para que el paquete pueda ser utilizados por aquellos que no tienen conocimientos del lenguaje de programación.
