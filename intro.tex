%% Los cap'itulos inician con \chapter{T'itulo}, estos aparecen numerados y
%% se incluyen en el 'indice general.
%%
%% Recuerda que aqu'i ya puedes escribir acentos como: 'a, 'e, 'i, etc.
%% La letra n con tilde es: 'n.


\chapter{Introducción}

\todo{This is a note}

A lo largo de la historia de la agricultura, el hombre ha desarrollado el mejoramiento vegetal (o fitomejoramiento) en forma sistemática y lo ha convertido en un instrumento esencial para la mejora de la producción agrícola en términos de cantidad, calidad y diversidad.  

El fitomejoramiento, en un sentido amplio, busca alterar o modificar la herencia de las plantas para obtener cultivares  mejorados genéticamente, adaptados a condiciones específicas, de mayores rendimientos económicos y de mejor calidad que las variedades nativas o criollas \citep{Allard67}. En otras palabras, su objetivo es desarrollar plantas cuyo patrimonio hereditario esté de acuerdo con las condiciones, necesidades y recursos de los todos aquellos que producen, transforman y consumen productos vegetales. 

Las variedades mejoradas son el resultado del trabajo de desarrollo genético llevado a cabo en los programas de fitomejoramiento, los cuales se extienden a lo largo de varios años y requieren cuantiosas inversiones. Generalmente, en etapas tempranas de estos programas existe un gran número de genotipos experimentales con pocos antecedentes de evaluación; mientras que en etapas posteriores  se trabaja con pocos genotipos altamente selectos. En etapas avanzadas, los ensayos multiambientales (EMA), que comprenden experimentos en múltiples ambientes, son herramientas fundamentales para incrementar la productividad y rentabilidad de los cultivos. 

Como consecuencia de que los EMA se llevan a cabo en múltiples ambientes (ya sea localidades o años), la aparición de la interacción genotipo ambiente (IGA) es inevitable debido a las variaciones en las condiciones climáticas y de suelo. La IGA es considerada por los fitomejoradores como el principal factor que limita la respuesta a la selección y, en general, la eficiencia de los programas de mejoramiento, por provocar respuestas altamente variables en los diferentes ambientes (Crossa et al., 1990; Cruz Medina, 1992; Kang y Magari, 1996).Gauch y Zobel (1996) explicaron la importancia de IGA como: “Si no hubiera interacción, una sola variedad de trigo (\emph{Triticum aestivum} L.) o maíz (\emph{Zea mays} L.) o cualquier otro cultivo rendiría al máximo en todo el mundo, y además las variedades se deberían evaluar en un sólo lugar para proporcionar resultados universales".



Peto (1982) ha distinguido las interacciones cuantitativas, conocida también como interacción sin cambio de rango (NCOI) o \emph{no crossover}, de las interacciones cualitativas, denominada también con
cambio de rango (COI) o \emph{crossover} (Cornelius et al., 1996). Cuando dos genotipos X e Y tienen una respuesta diferencial en dos ambientes diferentes, pero su ordenación permanece sin cambios se dice que la IGA es \emph{no crossover} (Figura \ref{fig:fig11}(A)). Sin embargo, es de tipo \emph{crossover} cuando hay cambios en el orden de los genotipos (Figura  \ref{fig:fig11}(B)). Cuando los genotipos responden de manera similar en ambos ambientes (Figura \ref{fig:fig11}(C)) no hay IGA. 


\begin{figure}[h]
\begin{center}
\includegraphics[width=14cm]{./Graficos/interac}
\end{center}
\caption{Representación gráfica de tipos de IGA: (A)IGA no crossover, (B) IGA crossover y (C) no IGA}
\label{fig:fig11}
\end{figure}


Entre las implicancias de IGA en los programas de mejoramiento se encuentra por ejemplo el impacto negativo en la heredabilidad, cuanto menor sea la heredabilidad de un caracter, mayor será la dificultad para mejorar ese caracter mediante la selección. Como consecuencia, información sobre la estructura y la naturaleza de la IGA es particularmente útil para determinar, por ejemplo, si se deben desarrollar cultivares para todos los ambientes de interés o si se deberían desarrollar para ambientes específicos (Bridges, 1989). Además, distintos conceptos como regiones ecológicas, ecotipos, mega-ambientes, adaptaciones de germoplasma tanto en sentido amplio (a través de los ambientes) como específico (para cada ambiente o grupos de ambiente particular) (Kang et al., 2004) se pueden analizar a partir de la IGA (Yan y Hunt, 2001).


La vigencia comercial de las variedades puede extenderse durante varias décadas, por lo que su elección es crítica para que el productor evite pérdidas económicas por malas campañas y el suministro al mercado sea constante. Consecuentemente, un análisis adecuado de la información de los EMA es indispensable para que el programa de mejoramiento de los cultivos sea eficaz. El rendimiento medio en los ambientes es un indicador suficiente del rendimiento genotípico solo en ausencia de IGA (Yan y Kang, 2003). Sin embargo, la aparición de IGA es inevitable y no basta con la comparación de las medias de los genotipos, sino que se debe recurrir a una metodología estadística más aporopiada. La metodología estadística más difundida para analizar los datos provenientes de EMA se basa en modificaciones de los modelos de regresión, análisis de variancia (\emph{Analysis of Variance}, ANOVA) y técnicas de análisis multivariado. 

Particularmente, para el estudio de la IGA y los análisis que de ella se derivan, dos modelos multiplicativos han aumentado su popularidad entre los fitomejoradores como una herramienta de análisis gráfico: el modelo de los efectos principales aditivos e interacción multiplicativa (\emph{Additive Main effects and Multiplicative Interaction}, AMMI) (Kempton 1984, Gauch, 1988), y el de regresión por sitio (\emph{Site Regression model}, SREG) (Cornelius et al., 1996; Crossa y Cornelius, 1997 y 2002).  Estos modelos combinan un ANOVA con la descomposición de valores singulares (DVS) o un análisis de componentes principales (ACP) sobre la matriz residual de ANOVA. En SREG, el ANOVA se realiza sobre el efecto principal de A mientras que en AMMI se consideran los  efectos de G y A. Mientras que a través del modelo AMMI se obtiene el gráfico biplot \emph{Genotipe-Environment} (GE), el cual es usado para explorar patrones puramente atribuibles a los efectos la IGA, para el modelo SREG el biplot GGE es usado para explorar simultáneamente patrones de variación conjunta de G e IGA (Yan y Hunt (2002)).

Una limitación importante de la mayoría de las propuestas de análisis provenientes de EMA es que requieren que el conjunto de datos este completo. Aunque los EMA están diseñados para que todos los genotipos se evalúen en todos los ambientes,  las tablas de datos genotipo $\times$ ambiente completas son poco frecuentes (no todos los genotipos se encuentran en todos los ambientes). Esto ocurre, por ejemplo, debido a errores de medición o causas naturales como la destrucción de plantas por animales, inundaciones o durante la cosecha, la incorporación de nuevos genotipos y a que otros se descartan por su pobre desempeño (Hill y Rosemberg, 1985). En estos casos, entre las posibles soluciones para tratar con una tabla de datos incompleta se encuentran (i) el uso de un subconjunto completo de datos, eliminando aquellos genotipos que tienen valores faltantes (Ceccarelli et al., 2007, Yan et al., 2011), (ii) completar datos faltantes con la media ambiental, o (iii) imputación de datos faltantes con valores estimados utilizando, por ejemplo, un modelo multiplicativo (Kumar et al., 2012). 



En este contexto, el análisis de datos provenientes de EMA requiere metodología estadística sofisticada cuyas rutinas informáticas se encuentran disponibles en programas desarrollados por diferentes empresas. Esto genera el inconveniente de tener que disponer de todos los programas necesarios para los distintos análisis, atender los requerimientos de formatos de datos usados por cada uno, y comprender los diversos tipos de salidas en las que se ofrecen los resultados obtenidos. Además, algunos procedimientos, especialmente aquellas metodologías recientes, no se encuentran disponibles, y los costos de las licencias de dichos programas resultan muy elevados. 

El software R se trata de un proyecto de software libre distribuido bajo los términos de la \emph{General Public Licence} (GNU), desarrollado por \emph{The R Foundation for Statistical Computing}. Surge como resultado de la implementación de uno de los lenguajes más utilizados en investigación por la comunidad estadística, el lenguaje S. A diferencia de los programas estadísticos utilizados frecuentemente, R es un lenguaje de programación y no dispone de una interfaz gráfica en la cual los distintos análisis se realizan por menús, lo cual genera dificultad en su uso para aquéllos que no se encuentran familiarizados con el lenguaje de programación. Sin embargo,  por ser un software sea libre,  permite a los usuarios definir sus propias funciones dando lugar mayores posibilidades en cuanto a la manipulación y análisis de los datos que desea realizar. Si bien la versión básica del programa dista mucho de ser amigable, RStudio permite una interacción más fluida con el programa, actuando como una interfaz amigable con el usuario.  RStudio es un entorno de desarrollo integrado (IDE) gratuito y de código abierto para R. 

R forma parte de un proyecto colaborativo ya que promueve el hecho de que los usuarios compartan con la comunidad las funciones creadas por ellos, es decir que está en continuo desarrollo y actualización. A menudo, no resulta sencillo reutilizar una función creada por algún usuario, por ello, se ha introducido la posibilidad de crear paquetes (\emph{package}) o librerías. Éstas son una colección de objetos creados y organizados siguiendo un protocolo fijo que garantiza un soporte mínimo para el usuario así como la ausencia de errores (de sintaxis) en la programación.

Actualmente, R cuenta con 14 paquetes básicos y 29 recomendados para su funcionamiento instalados automaticamente en él, como por ejemplo, \emph{base} o \emph{stats}. La comunidad de usuarios de R ha ido creciendo notablemente en los últimos años, por lo que se ha incrementado la cantidad de paquetes que extienden las funciones básicas de R. Entre ellos se encuentran, \emph{plyr}, \emph{lubridate}, \emph{reshape2} y \emph{stringr} para la manipulación de los datos; \emph{ggplot2} y \emph{rgl} para la visualización; \emph{knitr} y \emph{xtable} para la presentación de resultados; entre otros. La lista completa de los paquetes oficiales puede consultarse en CRAN\footnote{CRAN (Comprehensive R Archve Network) es el repositorio oficial de paquetes de R, el lugar donde se publican las nuevas versiones del programa, etc. Contiene la lista completa de paquetes oficiales. \url{https://cran.r-project.org/web/packages/available_packages_by_name.html}}, se contaba con más de 14.000 paquetes disponibles en CRAN hasta junio de 2019. Esta gran variedad de paquetes es una de las razones de la gran difusión de R, cada usuario puede resolver el problema en el cual se encuentra trabajando utilizando el paquete de otro. Además de los paquetes oficiales, existen otros que pueden instalarse desde repositorios como, por ejemplo, Github. Sin embargo, no es sencillo encontrar un paquete que puede ser útil para un determinado fin sino que se debe recurrir a varios de ellos para cumplir un determinado objetivo. 

Con frecuencia, los mejoradores usan programas que tienen una interfaz para realizar el análisis estadístico deseado sin necesidad del manejo de un lenguaje de programación. En el año 2012 se creó el paquete \emph{Shiny} de R que permite desarrollar aplicaciones Web utilizando R, acercando la potencia de R a todo tipo de usuarios, sin tener que programar.

El objetivo del presente trabajo es: (i) crear un paquete de R que integre las funciones que permitan analizar los datos provenientes de EMA, incluyendo además metodología recientemente publicada que no se encuentra disponible en R; (ii) crear una interfaz gráfica, entre R y el usuario, mediante Shiny con el fin de poder realizar los análisis disponibles en el paquete creado sin necesidad de utilizar el lenguaje de programación.



Elousa, Paula. 2009. “¿EXISTE VIDA MÁS ALLÁ DEL SPSS? DESCUBRE R.” Revista Psicothema 21 (4): 652–55. http://www.psicothema.com/psicothema.asp?id=3686.

FSF. 2019. “¿Qué Es El Software Libre?” Free Software Foundation. https://www.fsf.org/es/recursos/que-es-el-software-libre.