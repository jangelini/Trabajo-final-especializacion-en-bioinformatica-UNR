\chapter{Conclusión}

En etapas avanzadas de los programas de mejoramiento vegetal, comúnmente se llevan a cabo EMA que consisten en la evaluación de un conjunto de variedades en múltiples ambientes. Un análisis adecuado de la información de los EMA es indispensable para el éxito del programa de mejoramiento genético de los cultivos. Dado que, la metodología utilizada se encuentra en constante desarrollo, muchas de ellas no se encuentran disponibles en programas comerciales. 

En este trabajo se muestra un flujo de trabajo reproducible para la construcción de paquetes de R que puede utilizarse de ejemplo para el desarrollo de nuevos paquetes. En particular se creó el paquete de R \emph{geneticae} que es de gran utildad para el análisis de datos provenientes de EMA por incluir metodología recientemente publicada además de la reunir las funciones más útiles para tal fin. En el momento de la escritura de este informe, pasaron 3 semanas desde su publicación en el repositorio CRAN y cuenta con más de 400 descargas, a pesar de que aún no se ha hecho difusión del paquete. 

Por otro lado, dado que el uso del software puede resultar dificultoso para aquellos analistas no familiarizados con la programación, se crea una aplicación web Shiny de libre acceso mediante conexión a internet que permite realizar los principales análisis implementados en el paquete sin necesidad de escribir líneas de código.

Se plantea para un futuro, continuar con la inclusión de metodología frecuentemente utilizada así como aquellas que se vayan desarrollando en el contexto de datos provenientes de EMA tanto en el paquete como en la aplicación web Shiny. 




