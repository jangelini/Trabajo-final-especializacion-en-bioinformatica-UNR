\chapter{Conclusión}

Como resultados del presente trabajo fue posible:\\

\textbf{Mostrar un flujo de trabajo reproducible para la construcción de paquetes de R}. El mismo se puede utilizar de ejemplo para el desarrollo de nuevos paquetes o imitar la construcción del paquete \emph{geneticae} objeto de este trabajo. \\

\textbf{Construir un paquete de R llamado \emph{geneticae}} para el análisis de datos provenientes de EMA. El mismo es de gran utilidad ya que incluye metodología recientemente publicada además de la reunir las herramientas más utilizadas por los fitomejoradores para el análisis gráfico. En el momento de la escritura de este informe, pasaron 3 semanas desde su publicación en el repositorio CRAN y cuenta con más de 400 descargas, a pesar de que aún no se ha hecho difusión del paquete. \\

\textbf{Desarrollar una aplicación web Shiny denominada \emph{Geneticae}}, la cual es de suma importancia para aquellos analistas no familiarizados con la programación. Esta es de libre acceso mediante conexión a internet que permite realizar los principales análisis implementados en el paquete sin necesidad de escribir líneas de código. \\

Se plantea como línea futura, continuar con la inclusión de los avances metodologicos que se vayan publicando en el contexto de datos provenientes de EMA tanto en el paquete como en la aplicación web Shiny. 




