%% Los cap'itulos inician con \chapter{T'itulo}, estos aparecen numerados y
%% se incluyen en el 'indice general.
%%
%% Recuerda que aqu'i ya puedes escribir acentos como: 'a, 'e, 'i, etc.
%% La letra n con tilde es: 'n.
\chapter{Resultados}
\section{Paquete de R \emph{geneticae}}

El paquete \emph{geneticae} ofrece funciones para el análisis de datos de etapas avanzadas de programas de mejoramiento, donde se evalúan pocos genotipos. 

Una vez que se instalado el paquete geneticae, se debe cargar en la sesion de R mediante el comando: \textcolor{blue}{library}(geneticae)

Es posible obtener información detallada sobre las funciones del paquete geneticae mediante de los archivos de ayuda indicando \textcolor{blue}{help}(package = "geneticae").  La ayuda para una función, por ejemplo, \textcolor{blue}{imputation}(), en una sesión R se puede obtener usando \emph{?imputation} o \textcolor{blue}{help}(imputation).


\subsection{Conjuntos de datos en geneticae}

El paquete geneticae proporciona dos conjuntos de datos para ilustrar la metodología incluida para analizar los datos MET.

\begin{itemize}
\item yan.winterwheat dataset: rendimiento de 18 variedades de trigo de invierno cultivadas en nueve ambientes en Ontario en 1993. No hay réplicas disponibles en los datos. Este conjunto de datos se obtuvo del paquete agridat.
\end{itemize}
\begin{lstlisting}
data(yan.winterwheat)
dat_yan <- yan.winterwheat
head(dat_yan)
\end{lstlisting}

\begin{verbatim}
##   gen  env yield
## 1 Ann BH93 4.460
## 2 Ari BH93 4.417
## 3 Aug BH93 4.669
## 4 Cas BH93 4.732
## 5 Del BH93 4.390
## 6 Dia BH93 5.178
\end{verbatim}
\begin{itemize}
\item plrv dataset: rendimiento, peso de planta y parcela de 28 clones de la población del virus del enrollamiento de la papa (PLRV) evaluada en seis entornos. Las réplicas están disponibles en los datos. Este conjunto de datos se obtuvo del paquete agricolae.
\end{itemize}
\begin{lstlisting}
data(plrv)
dat_rep <- plrv
head(dat_rep)
\end{lstlisting}


\begin{verbatim}
##   Genotype Locality Rep WeightPlant WeightPlot    Yield
## 1   102.18     Ayac   1   0.5100000       5.10 18.88889
## 2   104.22     Ayac   1   0.3450000       2.76 12.77778
## 3   121.31     Ayac   1   0.5425000       4.34 20.09259
## 4   141.28     Ayac   1   0.9888889       8.90 36.62551
## 5   157.26     Ayac   1   0.6250000       5.00 23.14815
## 6    163.9     Ayac   1   0.5120000       2.56 18.96296
\end{verbatim}

 
\subsection{Funciones en geneticae}

\textbf{Modelo de regresión por sitio}

Para ejecutar la función \textcolor{blue}{GGEmodel}(), se debe proporcionar un conjunto de datos con genotipos, ambientes, repeticiones (si hay disponibles), el fenotipo observado y los nombres que dichas variables tienen en el archivo de entrada. Además, se debe indicar el método de centrado, escala y SVD.

Cuando no hay repeticiones disponibles en el conjunto de datos, como es el caso del conjunto de datos yan.winterwheat, el modelo GGE se indica de la siguiente manera:


\begin{lstlisting}
GGE1 <- GGEmodel(dat_yan, genotype = "gen", environment = "env", response = "yield", centering = "tester")
\end{lstlisting}


Sin embargo, en el caso de que haya repeticiones disponibles, como el conjunto de datos plrv, se indica de la siguiente manera:


\begin{lstlisting}
GGE1_rep <- GGEmodel(dat_rep, genotype = "Genotype", environment = "Locality", response = "Yield", rep = "Rep", centering = "tester")
\end{lstlisting}


La salida de la función \textcolor{blue}{GGEmodel}() es una lista con los siguientes elementos:


\begin{itemize}
\item coordgenotype: trazado de coordenadas para genotipos de todos los componentes.
\item coordenviroment: trazado de coordenadas para entornos de todos los componentes.
\item valores propios: vector de valores propios de cada componente.
\item vartotal: varianza general.
\item varexpl: porcentaje de varianza explicado por cada componente.
\item labelgen: nombres de genotipo.
\item labelenv: nombres de entorno.
\item ejes: etiquetas de eje.
\item Datos: datos de entrada escalados y centrados.
\item centrado: nombre del método de centrado.
\item escala: nombre del método de escala.
\item SVP: nombre del método SVP.
\end{itemize}


Por ejemplo, para el conjunto de datos yan.winterwheat:


\begin{lstlisting}
names(GGE1)
\end{lstlisting}

\begin{verbatim}
##  [1] "coordgenotype"   "coordenviroment" "eigenvalues"    
##  [4] "vartotal"        "varexpl"         "labelgen"       
##  [7] "labelenv"        "labelaxes"       "Data"           
## [10] "centering"       "scaling"         "SVP"
\end{verbatim}


\textbf{Biplot GGE}

Para ejecutar la función \textcolor{blue}{GGEPlot}(), se requiere un objeto de la clase \textcolor{blue}{GGEmodel}(). La salida es un biplot construido a través de los componentes principales generados por \textcolor{blue}{GGEmodel}().

Los diferentes biplots que se pueden obtener usando la función \textcolor{blue}{GGEPlot}() se muestran usando el conjunto de datos yan.winterwheat. Si hay repeticiones disponibles en el conjunto de datos, como es el caso del conjunto plrv, se debe indicar el nombre de la columna que contiene las réplicas en el archivo de entrada.


\begin{itemize}
\item Biplot básico

\begin{lstlisting}
GGEPlot(GGE1, type = "Biplot")
\end{lstlisting}

\begin{figure}[H]
	\begin{center}
		\includegraphics[width=10cm]{./Graficos/GGE_BIPLOT.png}
	\end{center}
	\caption{Biplot básico obtenido de la función \textcolor{blue}{GGEPlot}()}
\end{figure}

\item Ranking de los cultivares en función de su rendimiento en el ambiente OA93.

\begin{lstlisting}
GGEPlot(GGE1, type = "Selected Environment", selectedE = "OA93")
\end{lstlisting}


\begin{figure}[H]
	\begin{center}
		\includegraphics[width=10cm]{./Graficos/SelectedEnvironment.png}
	\end{center}
	\caption{Ranking de cultivares para un ambiente determinado obtenido de la función \textcolor{blue}{GGEPlot}()}
\end{figure}


\item Ranking de los ambientes en función del rendimiento relativo del cultivar Kat.

\begin{lstlisting}
GGEPlot(GGE1, type = "Selected Genotype", selectedG = "Kat")
\end{lstlisting}

\begin{figure}[H]
	\begin{center}
		\includegraphics[width=10cm]{./Graficos/SelectedGenotype.png}
	\end{center}
	\caption{Ranking de ambientes para cultivar determinado obtenido de la función \textcolor{blue}{GGEPlot}()}
\end{figure}


\item Relación entre ambientes.

\begin{lstlisting}
GGEPlot(GGE1, type = "Relationship Among Environments")
\end{lstlisting}

\begin{figure}[H]
	\begin{center}
		\includegraphics[width=10cm]{./Graficos/RelationshipAmongEnvironments.png}
	\end{center}
	\caption{Relación entre ambientes obtenido de la función \textcolor{blue}{GGEPlot}()}
\end{figure}

\item Comparación entre los genotipos Kat y Cas.

\begin{lstlisting}
GGEPlot(GGE1, type = "Comparison of Genotype", selectedG1 = "Kat", selectedG2 = "Cas")
\end{lstlisting}

\begin{figure}[H]
	\begin{center}
		\includegraphics[width=10cm]{./Graficos/ComparisonofGenotype.png}
	\end{center}
	\caption{Comparación entre dos genotipos obtenido de la función \textcolor{blue}{GGEPlot}()}
\end{figure}


\item Identificación del mejor cultivar en cada ambiente.

\begin{lstlisting}
GGEPlot(GGE1, type = "Which Won Where/What")
\end{lstlisting}

\begin{figure}[H]
	\begin{center}
		\includegraphics[width=12cm]{./Graficos/WhichWonWhereWhat.png}
	\end{center}
	\caption{Identificación del mejor cultivar en cada ambiente a partir de la función \textcolor{blue}{GGEPlot}()}
\end{figure}



\item Evaluación de los ambientes basados tanto en la capacidad de discriminación como en la representatividad.

\begin{lstlisting}
GGEPlot(GGE1, type = "Discrimination vs. representativeness")
\end{lstlisting}

\begin{figure}[H]
	\begin{center}
		\includegraphics[width=12cm]{./Graficos/Discriminationvsrepresentativeness.png}
	\end{center}
	\caption{Evaluación de los ambientes basados tanto en la capacidad de discriminación y representatividad a partir de la función \textcolor{blue}{GGEPlot}()}
\end{figure}



\item Clasificación de ambientes con respecto al ambiente ideal.

\begin{lstlisting}
GGEPlot(GGE1, type = "Ranking Environments")
\end{lstlisting}

\begin{figure}[H]
	\begin{center}
		\includegraphics[width=12cm]{./Graficos/RankingEnvironments.png}
	\end{center}
	\caption{Clasificación de ambientes con respecto al ambiente ideal a partir de la función \textcolor{blue}{GGEPlot}()}
\end{figure}


\item Clasificación de genotipos con respecto al genotipo ideal.

\begin{lstlisting}
GGEPlot(GGE1, type = "Ranking Genotypes")
\end{lstlisting}

\begin{figure}[H]
	\begin{center}
		\includegraphics[width=10cm]{./Graficos/RankingGenotypes.png}
	\end{center}
	\caption{Clasificación de genotipos con respecto al genotipo ideal a partir de la función \textcolor{blue}{GGEPlot}()}
\end{figure}

\item Evaluación de los cultivares con base en el rendimiento promedio y la estabilidad.

\begin{lstlisting}
GGEPlot(GGE1, type = "Mean vs. Stability")
\end{lstlisting}

\begin{figure}[H]
	\begin{center}
		\includegraphics[width=10cm]{./Graficos/MeanvsStability.png}
	\end{center}
	\caption{Evaluación de los cultivares con base en el rendimiento promedio y la estabilidad a partir de la función \textcolor{blue}{GGEPlot}()}
\end{figure}

\end{itemize}


\textbf{Classic AMMI model}

Para ejecutar la función \textcolor{blue}{rAMMI}(), como en la función \textcolor{blue}{GGEmodel}(), se debe proporcionar un conjunto de datos con genotipo, entorno, repeticiones (si las hay) y la variable de respuesta. Se debe indicar el nombre de las columnas que contienen cada una de estas variables en el conjunto de datos de entradas. La salida de la función es un biplot.

A continuación se muestra el biplot GE obtenido del modelo AMMI clásico obtenido con el conjunto de datos yan.winterwheat.

\begin{lstlisting}
rAMMI(dat_yan, genotype = "gen", environment = "env", response = "yield", type = "AMMI")
\end{lstlisting}

\begin{figure}[H]
	\begin{center}
		\includegraphics[width=10cm]{./Graficos/AMMI.png}
	\end{center}
	\caption{Biplot GE obtenido del modelo clasico AMMI}
\end{figure}

\textbf{Robust AMMI model}

Como se dijo anteriormente, el modelo AMMI clasico, en su forma estándar, no funciona bien en presencia de observaciones atípicas. Dado que los outliers son muy comun en los datos agronómicos, Rodrigues et al. (2015) proponen cinco modelos AMMI robustos, que permiten superar el problema de la contaminación de datos con observaciones atípicas. Los biplots de los cinco modelos AMMI robustos propuestos por Rodrigues et al. (2015), se pueden obtener utilizando la función \textcolor{blue}{rAMMI}() A continuación se muestran los biplots obtenidos con dichos modelos robustos usando el conjunto de datos yan.winterwheat.

\begin{itemize}

\item  modelo "rAMMI"

\begin{lstlisting}
rAMMI(dat_yan, genotype = "gen", environment = "env", response = "yield", type = "rAMMI")
\end{lstlisting}

\begin{figure}[H]
	\begin{center}
		\includegraphics[width=10cm]{./Graficos/rAMMI.png}
	\end{center}
	\caption{Biplot GE obtenido del modelo robusto rAMMI}
\end{figure}


\item  modelo "hAMMI"

\begin{lstlisting}
rAMMI(dat_yan, genotype = "gen", environment = "env", response = "yield", type = "hAMMI")
\end{lstlisting}

\begin{figure}[H]
	\begin{center}
		\includegraphics[width=10cm]{./Graficos/hAMMI.png}
	\end{center}
	\caption{Biplot GE obtenido del modelo robusto hAMMI}
\end{figure}


\item  modelo "gAMMI"

\begin{lstlisting}
rAMMI(dat_yan, genotype = "gen", environment = "env", response = "yield", type = "gAMMI")
\end{lstlisting}

\begin{figure}[H]
	\begin{center}
		\includegraphics[width=10cm]{./Graficos/gAMMI.png}
	\end{center}
	\caption{Biplot GE obtenido del modelo robusto gAMMI}
\end{figure}



\item  modelo "lAMMI"

\begin{lstlisting}
rAMMI(dat_yan, genotype = "gen", environment = "env", response = "yield", type = "lAMMI")
\end{lstlisting}


\begin{figure}[H]
	\begin{center}
		\includegraphics[width=10cm]{./Graficos/lAMMI.png}
	\end{center}
	\caption{Biplot GE obtenido del modelo robusto lAMMI}
\end{figure}


\item  modelo "ppAMMI"
\begin{lstlisting}
rAMMI(dat_yan, genotype = "gen", environment = "env", response = "yield", type = "ppAMMI")
\end{lstlisting}


\begin{figure}[H]
	\begin{center}
		\includegraphics[width=10cm]{./Graficos/ppAMMI.png}
	\end{center}
	\caption{Biplot GE obtenido del modelo robusto ppAMMI}
\end{figure}

\end{itemize}

\textbf{Métodos de imputación}
Una limitación importante de los modelos presentados anteriormente es que requieren una que el conjunto de datos este completo. Por lo tanto, en el paquete se incluyen una serie de metodologías propuestas, algunas de las cuales no se encuentran disponible en R, para superar el problema de falta de equilibrio. 

El conjunto de datos yan.winterwheat se utilizó como ejemplo. Como el conjunto de datos no contaba con observaciones perdidas, algunas fueron eliminadas con el objetivo de mostrar las metodologías de imputación incluidas.

\begin{lstlisting}
# generates missing data
dat_yan[1, 3] <- NA
dat_yan[3, 3] <- NA
dat_yan[2, 3] <- NA
\end{lstlisting}


\begin{itemize}
\item GabrielEigein proposed by Arciniegas-Alarcón S., et al. (2010).
\end{itemize}
\begin{lstlisting}
imputation(dat_yan, PC.nb = 2, genotype = "gen", environment = "env", response = "yield", type = "EM-AMMI")
\end{lstlisting}
\centering
\begin{small}
\begin{Verbatim}[frame=single,baselinestretch=0.3]
##         BH93  EA93  HW93  ID93  KE93  NN93  OA93  RN93  WP93
## Ann 4.150120 4.150 2.849 3.084 5.940 4.450 4.351 4.039 2.672
## Ari 4.035814 4.771 2.912 3.506 5.699 5.152 4.956 4.386 2.938
## Aug 4.305244 4.578 3.098 3.460 6.070 5.025 4.730 3.900 2.621
## Cas 4.732000 4.745 3.375 3.904 6.224 5.340 4.226 4.893 3.451
## Del 4.390000 4.603 3.511 3.848 5.773 5.421 5.147 4.098 2.832
## Dia 5.178000 4.475 2.990 3.774 6.583 5.045 3.985 4.271 2.776
## Ena 3.375000 4.175 2.741 3.157 5.342 4.267 4.162 4.063 2.032
## Fun 4.852000 4.664 4.425 3.952 5.536 5.832 4.168 5.060 3.574
## Ham 5.038000 4.741 3.508 3.437 5.960 4.859 4.977 4.514 2.859
## Har 5.195000 4.662 3.596 3.759 5.937 5.345 3.895 4.450 3.300
## Kar 4.293000 4.530 2.760 3.422 6.142 5.250 4.856 4.137 3.149
## Kat 3.151000 3.040 2.388 2.350 4.229 4.257 3.384 4.071 2.103
## Luc 4.104000 3.878 2.302 3.718 4.555 5.149 2.596 4.956 2.886
## Reb 4.375000 4.701 3.655 3.592 6.189 5.141 3.933 4.208 2.925
## Ron 4.940000 4.698 2.950 3.898 6.063 5.326 4.302 4.299 3.031
## Rub 3.786000 4.969 3.379 3.353 4.774 5.304 4.322 4.858 3.382
## Zav 4.238000 4.654 3.607 3.914 6.641 4.830 5.014 4.363 3.111
## m12 3.340000 3.854 2.419 2.783 4.629 5.090 3.281 3.918 2.561
\end{Verbatim}
\end{small}


\begin{itemize}
\item EM-AMMI proposed by Gauch and Zobel (1990).
\end{itemize}
\begin{lstlisting}
imputation(dat_yan, PC.nb = 1, genotype = "gen", environment = "env", response = "yield", type = "EM-AMMI")
\end{lstlisting}

\begin{verbatim}
##         BH93  EA93  HW93  ID93  KE93  NN93  OA93  RN93  WP93
## Ann 4.136249 4.150 2.849 3.084 5.940 4.450 4.351 4.039 2.672
## Ari 4.474249 4.771 2.912 3.506 5.699 5.152 4.956 4.386 2.938
## Aug 4.386299 4.578 3.098 3.460 6.070 5.025 4.730 3.900 2.621
## Cas 4.732000 4.745 3.375 3.904 6.224 5.340 4.226 4.893 3.451
## Del 4.390000 4.603 3.511 3.848 5.773 5.421 5.147 4.098 2.832
## Dia 5.178000 4.475 2.990 3.774 6.583 5.045 3.985 4.271 2.776
## Ena 3.375000 4.175 2.741 3.157 5.342 4.267 4.162 4.063 2.032
## Fun 4.852000 4.664 4.425 3.952 5.536 5.832 4.168 5.060 3.574
## Ham 5.038000 4.741 3.508 3.437 5.960 4.859 4.977 4.514 2.859
## Har 5.195000 4.662 3.596 3.759 5.937 5.345 3.895 4.450 3.300
## Kar 4.293000 4.530 2.760 3.422 6.142 5.250 4.856 4.137 3.149
## Kat 3.151000 3.040 2.388 2.350 4.229 4.257 3.384 4.071 2.103
## Luc 4.104000 3.878 2.302 3.718 4.555 5.149 2.596 4.956 2.886
## Reb 4.375000 4.701 3.655 3.592 6.189 5.141 3.933 4.208 2.925
## Ron 4.940000 4.698 2.950 3.898 6.063 5.326 4.302 4.299 3.031
## Rub 3.786000 4.969 3.379 3.353 4.774 5.304 4.322 4.858 3.382
## Zav 4.238000 4.654 3.607 3.914 6.641 4.830 5.014 4.363 3.111
## m12 3.340000 3.854 2.419 2.783 4.629 5.090 3.281 3.918 2.561
\end{verbatim}



\begin{itemize}
\item EM-SVD proposed by Perry (2009)
\end{itemize}
\begin{lstlisting}
imputation(dat_yan, genotype = "gen", environment = "env", response = "yield", type = "EM-SVD")
\end{lstlisting}

\begin{verbatim}
##           [,1]  [,2]  [,3]  [,4]  [,5]  [,6]  [,7]  [,8]  [,9]
##  [1,] 4.332467 4.150 2.849 3.084 5.940 4.450 4.351 4.039 2.672
##  [2,] 4.332467 4.771 2.912 3.506 5.699 5.152 4.956 4.386 2.938
##  [3,] 4.332467 4.578 3.098 3.460 6.070 5.025 4.730 3.900 2.621
##  [4,] 4.732000 4.745 3.375 3.904 6.224 5.340 4.226 4.893 3.451
##  [5,] 4.390000 4.603 3.511 3.848 5.773 5.421 5.147 4.098 2.832
##  [6,] 5.178000 4.475 2.990 3.774 6.583 5.045 3.985 4.271 2.776
##  [7,] 3.375000 4.175 2.741 3.157 5.342 4.267 4.162 4.063 2.032
##  [8,] 4.852000 4.664 4.425 3.952 5.536 5.832 4.168 5.060 3.574
##  [9,] 5.038000 4.741 3.508 3.437 5.960 4.859 4.977 4.514 2.859
## [10,] 5.195000 4.662 3.596 3.759 5.937 5.345 3.895 4.450 3.300
## [11,] 4.293000 4.530 2.760 3.422 6.142 5.250 4.856 4.137 3.149
## [12,] 3.151000 3.040 2.388 2.350 4.229 4.257 3.384 4.071 2.103
## [13,] 4.104000 3.878 2.302 3.718 4.555 5.149 2.596 4.956 2.886
## [14,] 4.375000 4.701 3.655 3.592 6.189 5.141 3.933 4.208 2.925
## [15,] 4.940000 4.698 2.950 3.898 6.063 5.326 4.302 4.299 3.031
## [16,] 3.786000 4.969 3.379 3.353 4.774 5.304 4.322 4.858 3.382
## [17,] 4.238000 4.654 3.607 3.914 6.641 4.830 5.014 4.363 3.111
## [18,] 3.340000 3.854 2.419 2.783 4.629 5.090 3.281 3.918 2.561
\end{verbatim}


\begin{itemize}
\item WGabriel proposed by Alarcon…..
\end{itemize}
\begin{lstlisting}
imputation(dat_yan, genotype = "gen", environment = "env", response = "yield", type = "WGabriel")
\end{lstlisting}

\begin{verbatim}
##         BH93  EA93  HW93  ID93  KE93  NN93  OA93  RN93  WP93
## Ann 4.004664 4.150 2.849 3.084 5.940 4.450 4.351 4.039 2.672
## Ari 4.455727 4.771 2.912 3.506 5.699 5.152 4.956 4.386 2.938
## Aug 4.328095 4.578 3.098 3.460 6.070 5.025 4.730 3.900 2.621
## Cas 4.732000 4.745 3.375 3.904 6.224 5.340 4.226 4.893 3.451
## Del 4.390000 4.603 3.511 3.848 5.773 5.421 5.147 4.098 2.832
## Dia 5.178000 4.475 2.990 3.774 6.583 5.045 3.985 4.271 2.776
## Ena 3.375000 4.175 2.741 3.157 5.342 4.267 4.162 4.063 2.032
## Fun 4.852000 4.664 4.425 3.952 5.536 5.832 4.168 5.060 3.574
## Ham 5.038000 4.741 3.508 3.437 5.960 4.859 4.977 4.514 2.859
## Har 5.195000 4.662 3.596 3.759 5.937 5.345 3.895 4.450 3.300
## Kar 4.293000 4.530 2.760 3.422 6.142 5.250 4.856 4.137 3.149
## Kat 3.151000 3.040 2.388 2.350 4.229 4.257 3.384 4.071 2.103
## Luc 4.104000 3.878 2.302 3.718 4.555 5.149 2.596 4.956 2.886
## Reb 4.375000 4.701 3.655 3.592 6.189 5.141 3.933 4.208 2.925
## Ron 4.940000 4.698 2.950 3.898 6.063 5.326 4.302 4.299 3.031
## Rub 3.786000 4.969 3.379 3.353 4.774 5.304 4.322 4.858 3.382
## Zav 4.238000 4.654 3.607 3.914 6.641 4.830 5.014 4.363 3.111
## m12 3.340000 3.854 2.419 2.783 4.629 5.090 3.281 3.918 2.561
\end{verbatim}

\begin{itemize}
\item EM-PCA proposed by
\end{itemize}
\begin{lstlisting}
imputation(dat_yan, genotype = "gen", environment = "env", response = "yield", type = "EM-PCA")
\end{lstlisting}

\begin{verbatim}
##         BH93  EA93  HW93  ID93  KE93  NN93  OA93  RN93  WP93
## Ann 3.980317 4.150 2.849 3.084 5.940 4.450 4.351 4.039 2.672
## Ari 4.463093 4.771 2.912 3.506 5.699 5.152 4.956 4.386 2.938
## Aug 4.327731 4.578 3.098 3.460 6.070 5.025 4.730 3.900 2.621
## Cas 4.732000 4.745 3.375 3.904 6.224 5.340 4.226 4.893 3.451
## Del 4.390000 4.603 3.511 3.848 5.773 5.421 5.147 4.098 2.832
## Dia 5.178000 4.475 2.990 3.774 6.583 5.045 3.985 4.271 2.776
## Ena 3.375000 4.175 2.741 3.157 5.342 4.267 4.162 4.063 2.032
## Fun 4.852000 4.664 4.425 3.952 5.536 5.832 4.168 5.060 3.574
## Ham 5.038000 4.741 3.508 3.437 5.960 4.859 4.977 4.514 2.859
## Har 5.195000 4.662 3.596 3.759 5.937 5.345 3.895 4.450 3.300
## Kar 4.293000 4.530 2.760 3.422 6.142 5.250 4.856 4.137 3.149
## Kat 3.151000 3.040 2.388 2.350 4.229 4.257 3.384 4.071 2.103
## Luc 4.104000 3.878 2.302 3.718 4.555 5.149 2.596 4.956 2.886
## Reb 4.375000 4.701 3.655 3.592 6.189 5.141 3.933 4.208 2.925
## Ron 4.940000 4.698 2.950 3.898 6.063 5.326 4.302 4.299 3.031
## Rub 3.786000 4.969 3.379 3.353 4.774 5.304 4.322 4.858 3.382
## Zav 4.238000 4.654 3.607 3.914 6.641 4.830 5.014 4.363 3.111
## m12 3.340000 3.854 2.419 2.783 4.629 5.090 3.281 3.918 2.561
\end{verbatim}


\section{Geneticae Shiny Web App}

La aplicación Geneticae se organiza en las siguientes pestañas:
\begin{itemize}
\item Los datos
\item Análisis descriptivo
\item ANOVA
\item Biplot GGE
\item Biplot GE
\item Ayuda
\end{itemize}

En muchos casos, algunos atributos estilísticos de salida pueden personalizarse para que el usuario obtenga la salida a su gusto. A su vez, los gráficos obtenidos pueden ser descargados.

\subsection{Los datos}
Al iniciar la aplicación Geneticae, se muestra una pantalla en la cual se carga el conjunto de datos a analizar. La aplicación admite datos en formato .csv, delimitados por coma o punto y coma; y también acepta la primera fila como encabezado. La aplicación puede leer un tipo de formato de datos: 
\begin{itemize}
\item Cada fila contiene una observación, que se compone de tres o cuatro valores: nombre del cultivar, ambiente, repetición si \item está disponible y valor fenotipico medido.
\item La primera fila de encabezado contiene los nombres de cada variable. Los encabezados pueden dar cualquier nombre que elija, y deben indicarse al cargar el archivo de datos.
\item El número de repeticiones puede diferir con los genotipos y los entornos.
\end{itemize}

Se utilizan dos conjuntos de datos, incluidos en el paquete Geneticae, para ilustrar la aplicación. Estos conjuntos de datos, uno de los cuales tiene repeticiones (conjunto de datos plrv) y el otro no (conjunto de datos yang), los cuales se pueden ver y descargar en la pestaña \emph{The data} $\rightarrow$ \emph{Example datasets} (Figura \ref{fig:fig41},\ref{fig:fig42}). 

\begin{figure}[H]
	\begin{center}
		\includegraphics[width=17cm]{./Graficos/Exampledatasets_withoutrep.png}
	\end{center}
	\caption{Conjunto de datos sin repetición disponible en Shiny Web App}
	\label{fig:fig41}
\end{figure}


\begin{figure}[H]
	\begin{center}
		\includegraphics[width=17cm]{./Graficos/Exampledatasets_withrep.png}
	\end{center}
	\caption{Conjunto de datos sin repetición disponible en Shiny Web App}
	\label{fig:fig42}
\end{figure}

\subsection{Análisis descriptivo}

El menú estadística descriptiva le permite describir un conjunto de datos utilizando diagrama de caja (o \emph{boxplot}), gráfico y matriz de correlación y gráfico de interacción.

\subsubsection{\emph{Boxplot}}
El \emph{boxplot} proporciona una medida central, la mediana y una idea de la dispersión a través del rango y el rango intercuartil. La posición de la mediana dentro de la caja y la similitud en la longitud de los bigotes nos dan una idea de la simetría de la distribución. 

Un boxplot intetactivo que compara el caracter cuantitativo de interés a través de genotipos, así como a través de los ambientes se pueden obtener (Figura \ref{fig:fig43},\ref{fig:fig44}).  Estos gráficos se pueden descagar en formato interactivo (.HTML) a partir del boton \emph{Download} (Figura \ref{fig:fig43}), así como también en formato .png (Figura \ref{fig:fig44}).

\begin{figure}[H]
	\begin{center}
		\includegraphics[width=17cm]{./Graficos/Exampledatasets_withrep.png}
	\end{center}
	\caption{Boxplot de genotipos a través de los ambientes para el conjunto de datos Plrv}
	\label{fig:fig43}
\end{figure}


\begin{figure}[H]
	\begin{center}
		\includegraphics[width=17cm]{./Graficos/Exampledatasets_withrep.png}
	\end{center}
	\caption{Boxplot de ambientes a través de los genotipos para el conjunto de datos Plrv}
	\label{fig:fig44}
\end{figure}

\subsubsection{Gráfico de correlación}
El correlograma o gráfico de correlación muestra la correlación entre pares de variables. Se pueden mostrar las correlaciones de Pearson y Spearman. Las correlaciones positivas se muestran en azul y las negativas en rojo. La intensidad del color y el tamaño del círculo son proporcionales a los coeficientes de correlación. Se pueden trazar ambas correlaciones entre entornos y entre genotipos (Figura \ref{fig:fig45},\ref{fig:fig46}).


\begin{figure}[H]
	\begin{center}
		\includegraphics[width=17cm]{./Graficos/Exampledatasets_withrep.png}
	\end{center}
	\caption{Boxplot de genotipos a través de los ambientes para el conjunto de datos Plrv}
	\label{fig:fig45}
\end{figure}


\begin{figure}[H]
	\begin{center}
		\includegraphics[width=17cm]{./Graficos/Exampledatasets_withrep.png}
	\end{center}
	\caption{Boxplot de ambientes a través de los genotipos para el conjunto de datos Plrv}
	\label{fig:fig46}
\end{figure}

\subsubsection{Matriz de correlación}
Una matriz de correlación se utiliza como una forma de resumir datos. Muestra los coeficientes de correlación de pares de variables. Las correlaciones de Spearman o Pearson se pueden calcular tanto para entornos como para genotipos (Figura \ref{fig:fig47},\ref{fig:fig48}).


\begin{figure}[H]
	\begin{center}
		\includegraphics[width=17cm]{./Graficos/Exampledatasets_withrep.png}
	\end{center}
	\caption{Boxplot de genotipos a través de los ambientes para el conjunto de datos Plrv}
	\label{fig:fig47}
\end{figure}


\begin{figure}[H]
	\begin{center}
		\includegraphics[width=17cm]{./Graficos/Exampledatasets_withrep.png}
	\end{center}
	\caption{Boxplot de ambientes a través de los genotipos para el conjunto de datos Plrv}
	\label{fig:fig48}
\end{figure}

\subsubsection{Gráfico de interacción}
Un diagrama de interacción es una representación visual de la interacción entre los efectos de dos factores, o entre un factor y una variable numérica. 

Se puede obtener el gráfico interactivo que muestra el cambio en el efecto genotípico a través de los entornos y también el que muestra el cambio en el efecto ambiental a través de los genotipos (Figura \ref{fig:fig49},\ref{fig:fig410}). Es posible descargarlo en formato interactivo (.HTML) a partir del boton \emph{Download} (Figura \ref{fig:fig49}), así como también en formato .png como se muestra en la Figura \ref{fig:fig410}.


\begin{figure}[H]
	\begin{center}
		\includegraphics[width=17cm]{./Graficos/Exampledatasets_withrep.png}
	\end{center}
	\caption{Boxplot de genotipos a través de los ambientes para el conjunto de datos Plrv}
	\label{fig:fig49}
\end{figure}


\begin{figure}[H]
	\begin{center}
		\includegraphics[width=17cm]{./Graficos/Exampledatasets_withrep.png}
	\end{center}
	\caption{Boxplot de ambientes a través de los genotipos para el conjunto de datos Plrv}
	\label{fig:fig410}
\end{figure}


\subsection{Análisis de la variancia}


\subsection{Biplot GGE}
El biplot GGE aborda visualmente muchos problemas relacionados con la evaluación de los genotipo y ambientes de prueba. En el caso de repeticiones disponibles en el conjunto de datos, se obtiene el valor fenotípico promedio para cada combinación de genotipo y ambiente. Los valores faltantes no están permitidos.


\subsection{Biplot GE}

\subsection{Ayuda}