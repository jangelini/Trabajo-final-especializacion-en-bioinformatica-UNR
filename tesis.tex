\documentclass[oneside,numbers,spanish]{ezthesis}
\usepackage[utf8]{inputenc}
\usepackage{tcolorbox}
\usepackage{listings}
\usepackage{graphicx}
\usepackage{pdfpages}
\usepackage{lscape}
\usepackage{xcolor}
\usepackage{todonotes}
\usepackage{lipsum}
\usepackage{bbding}
\usepackage{pifont}
\usepackage{wasysym}
\usepackage{fancyvrb}
\usepackage{amssymb}
\usepackage{natbib}
\usepackage{bm}
\usepackage{color}    
\usepackage{enumitem}
\usepackage{float}
\tcbuselibrary{skins,xparse}
\usepackage{varwidth}
\usepackage{lipsum}

\usepackage{hyperref}
\hypersetup{
    colorlinks,
    citecolor=black,
    filecolor=black,
    linkcolor=black,
    urlcolor=black
}

\definecolor{fandango}{rgb}{0.71, 0.2, 0.54}
\definecolor{darkraspberry}{rgb}{0.53, 0.15, 0.34}
\definecolor{YellowGreen}{rgb}{0.31, 0.49, 0.16}
\definecolor{ForestGreen}{rgb}{0.31, 0.49, 0.16}
\definecolor{RoyalBlue}{rgb}{0.58,0,0.82}
\definecolor{mygray}{gray}{0.96}
\definecolor{backcolour}{rgb}{0.95,0.95,0.92}
%\definecolor{aurometalsaurus}{rgb}{0.43, 0.5, 0.5}
 \definecolor{aurometalsaurus}{rgb}{0.52, 0.52, 0.51}
\usepackage{titlesec}
\setlist[itemize]{noitemsep, topsep=0pt}

\setlength{\parskip}{4pt}
 
\makeatletter
\def\myaddcontentsline#1#2#3{%
  \addtocontents{#1}{\protect\contentsline{#2}{#3}{see \thesection\ at p. \thepage}{}}}
\renewcommand{\@todonotes@addElementToListOfTodos}{%
    \if@todonotes@colorinlistoftodos%
        \myaddcontentsline{tdo}{todo}{{%
            \colorbox{\@todonotes@currentbackgroundcolor}%
                {\textcolor{\@todonotes@currentbackgroundcolor}{o}}%
            \ \@todonotes@caption}}%
    \else%
        \myaddcontentsline{tdo}{todo}{{\@todonotes@caption}}%
    \fi}%
\newcommand*\mylistoftodos{%
  \begingroup
       \setbox\@tempboxa\hbox{see 9.9 at p. 99}%
       \renewcommand*\@tocrmarg{\the\wd\@tempboxa}%
       \renewcommand*\@pnumwidth{\the\wd\@tempboxa}%
       \listoftodos%
  \endgroup
}
\makeatother


\lstset{
basicstyle=\footnotesize,
keywordstyle=\bfseries,
showstringspaces=false,
backgroundcolor=\color{backcolour}, 
numbersep=4pt, 
columns = fullflexible,
tabsize=2,                      % sets default tabsize to 2 spaces
captionpos=b,                   % sets the caption-position to bottom
breaklines=true,                % sets automatic line breaking
mathescape = true,
breakatwhitespace=false,        % sets if automatic breaks should only happen at whitespace
language=R,
linewidth=0.95\columnwidth,
xleftmargin=0.5cm,
%frame = single,  
framexleftmargin=15pt,
keywordstyle=\color{blue},      % keyword style
commentstyle=\color{YellowGreen}   % comment style
%stringstyle=\color{ForestGreen}      % string literal style
}


\usepackage{tocbasic}

\DeclareTOCStyleEntry[
  entrynumberformat=\entrynumberwithprefix{\figurename},
  dynnumwidth,
  numsep=0.25em
]{tocline}{figure}
\newcommand\entrynumberwithprefix[2]{#1\enspace#2:\hfill}


%% # Opciones disponibles para el documento #
%%
%% Las opciones con un (*) son las opciones predeterminadas.
%%
%% Modo de compilar:
%%   draft            - borrador con marcas de fecha y sin im'agenes
%%   draftmarks       - borrador con marcas de fecha y con im'agenes
%%   final (*)        - version final de la tesis
%%
%% Tama'no de papel:
%%   letterpaper (*)  - tama'no carta (Am'erica)
%%   a4paper          - tama'no A4    (Europa)
%%
%% Formato de impresi'on:
%%   oneside          - hojas impresas por un solo lado
%%   twoside (*)      - hijas impresas por ambos lados
%%
%% Tama'no de letra:
%%   10pt, 11pt, o 12pt (*)
%%
%% Espaciado entre renglones:
%%   singlespace      - espacio sencillo
%%   onehalfspace (*) - espacio de 1.5
%%   doublespace      - a doble espacio
%%
%% Formato de las referencias bibliogr'aficas:
%%   numbers          - numeradas, p.e. [1]
%%   authoryear (*)   - por autor y a'no, p.e. (Newton, 1997)
%%
%% Opciones adicionales:
%%   spanish         - tesis escrita en espa'nol
%%
%% Desactivar opciones especiales:
%%   nobibtoc   - no incluir la bibiolgraf'ia en el 'Indice general
%%   nofancyhdr - no incluir "fancyhdr" para producir los encabezados
%%   nocolors   - no incluir "xcolor" para producir ligas con colores
%%   nographicx - no incluir "graphicx" para insertar gr'aficos
%%   nonatbib   - no incluir "natbib" para administrar la bibliograf'ia

%% Paquetes adicionales requeridos se pueden agregar tambi'en aqu'i.
%% Por ejemplo:
%\usepackage{subfig}
%\usepackage{multirow}

%% # Datos del documento #
%% Nota que los acentos se deben escribir: \'a, \'e, \'i, etc.
%% La letra n con tilde es: \~n.

\author{JULIA ANGELINI}
\title{Paquete de R y aplicación Web Shiny para el análisis de datos provenientes de ensayos multiambientales}
\degree{TRABAJO FINAL PARA OPTAR AL TÍTULO DE ESPECIALISTA EN BIOINFORMÁTICA}
\supervisor{DIRECTOR: Dr. Gerardo Cervigni \\
			CO-DIRECTOR: Mgs. Marcos Prunello}
\institution{UNIVERSIDAD NACIONAL DE ROSARIO}
\faculty{FACULTAD DE CIENCIAS AGRARIAS}
\submitdate{AÑO: 2021}
%% # M'argenes del documento #
%% 
%% Quitar el comentario en la siguiente linea para austar los m'argenes del
%% documento. Leer la documentaci'on de "geometry" para m'as informaci'on.

\geometry{top=30mm,bottom=30mm,inner=30mm,outer=20mm}

 
%% El siguiente comando agrega ligas activas en el documento para las
%% referencias cruzadas y citas bibliogr'aficas. Tiene que ser *la 'ultima*
%% instrucci'on antes de \begin{document}.
%\hyperlinking
\begin{document}

%% En esta secci'on se describe la estructura del documento de la tesis.
%% Consulta los reglamentos de tu universidad para determinar el orden
%% y la cantidad de secciones que debes de incluir.

%% # Portada de la tesis #
%% Mirar el archivo "titlepage.tex" para los detalles.
\begin{titlepage}
  \TitleBlock{
  \centering \includegraphics[height=3cm, keepaspectratio=true]{./Graficos/UNR.png}
  
  \vspace*{0.15cm}
  \normalsize \scshape \textbf{\insertfaculty} \\
  \vspace*{0.15cm}
  \textbf{\insertinstitution}
	}
  \TitleBlock[\vspace*{1.5cm}]{\large\scshape \textbf{\inserttitle}}
  \vspace*{0.55cm}

  \TitleBlock[\vspace*{1.5cm}]{\normalsize \scshape \textbf{\insertauthor}}
  \TitleBlock[\vspace*{0.75cm}]{\normalsize \scshape \textbf{\insertdegree} \\
  \rule{5cm}{0.2mm}}
  \TitleBlock[\vspace*{1.5cm}]{\normalsize \scshape 
  \textbf{\insertsupervisor}}
  \TitleBlock[\vspace*{1.5cm}]{\textbf{\insertsubmitdate}}
\end{titlepage}


%% # Prefacios #
%% Por cada prefacio (p.e. agradecimientos, resumen, etc.) crear
%% un nuevo archivo e incluirlo aqu'i.
%% Para m'as detalles y un ejemplo mirar el archivo "gracias.tex".
\pagestyle{plain}
\pagenumbering{roman}

\begin{center}
\textbf{\Large{Paquete de R y aplicación web Shiny para el análisis de datos provenientes de ensayos multiambientales}}
\end{center}

\vspace{1.5cm}

\begin{center}
Julia Angelini

\vspace{0.5cm}
Licenciada en Estadística – Universidad Nacional de Rosario
\end{center}
\vspace{1.5cm}
Este Trabajo Final es presentado como parte de los requisitos para optar al grado académico de Especialista en \textbf{Bioinformática}, de la Universidad Nacional de Rosario y no ha sido previamente presentada para la obtención de otro título en ésta u otra Universidad. El mismo contiene los resultados obtenidos en investigaciones llevadas a cabo en \textbf{el Centro de Estudios Fotosintéticos y Bioquímicos (CEFOBI)}, durante el período comprendido entre \textbf{los años 2017 y 2022}, bajo la dirección del \textbf{Dr. Gerardo Cervigni} y la co-dirección del \textbf{Mgs. Marcos Prunello}.  

\vspace{1.25cm}
Nombre y firma del autor\\
\vspace{0.05cm}
\hspace{0.95cm}Lic. Julia Angelini

\vspace{1.25cm}
Nombre y firma del Director\\
\vspace{0.05cm}
\hspace{0.95cm}Dr. Gerardo Cervigni
 
\vspace{1.25cm}
Nombre y firma del Co - Director\\
\vspace{0.05cm}
\hspace{1.3cm}Mgs. Marcos Prunello
\vspace{1.85cm}

\rightline{Defendida: \rule{3cm}{0.4pt} de 20\rule{1cm}{0.4pt}.}


%% Las secciones del "prefacio" inician con el comando \prefacesection{T'itulo}
%% Este tipo de secciones *no* van numeradas, pero s'i aparecen en el 'indice.
%%
%% Si quieres agregar una secci'on que no vaya n'umerada y que *tampoco*
%% aparesca en el 'indice, usa entonces el comando \chapter*{T'itulo}
%%
%% Recuerda que aqu'i ya puedes escribir acentos como: 'a, 'e, 'i, etc.
%% La letra n con tilde es: 'n.

\chapter*{Agradecimientos}
\begin{spacing}{1}

\emph{En este trabajo final, directa o indirectamente, participaron muchas personas a las que les quiero agradecer.}

\emph{En primer lugar al Dr. Gerardo Cervigni por haberme propuesto realizar la Especialización Bioinformática, compartir su conocimiento y experiencia a lo largo de todo el proceso, contagiando su pasión, entusiasmo y energía. }

\emph{Al Mgs. Marcos Prunello por acompañarme en el desarrollo del trabajo final, por su dedicación, sus consejos y su ejemplo que me incentiva a superarme como profesional. Sin su confianza, apoyo y atención, este trabajo no hubiera sido posible. No sólo me enriquecí en lo académico sino también con la amistad que pudimos forjar. }

\emph{Al Centro Computacional del Centro Científico Teconológico de Rosario, miembro del Sistema Nacional de Computación de Alto Rendimiento, por la predisposición, asesoramiento e instalación de los recursos adicionales necesarios para este trabajo. }

\emph{Al Dr. Sergio Arciniegas Alarcón por su predisposición en la inclusión de los avances metodológicos realizados por su equipo de investigación en este trabajo.}

\emph{A mis compañeros de la Especialización, por las largas horas de cursos, mates y almuerzos. En especial, a Jor y Lu, por el aliento en todo momento, por compartir excelentes momentos y porque gracias a la ayuda de ambas he podido entender cosas que no habría podido sola.}

\emph{A los docentes de la Especialización en Bioinformática por su dedicación y paciencia para enseñarle a alumnos provenientes de las más diversas áreas esta hermosa combinación entre Biología, Informática y Estadística.}

\emph{A mis padres por el amor y apoyo incondicional y por el esfuerzo  de  dedicar  sus  vidas  a  brindarnos  a mi hermano y a mí la  posibilidad  de construir nuestros futuros. A mi hermano, por su cariño, apoyo, acompañamiento y sentido del humor. A Otto, por su incomparable mezcla de amor y comprensión, por darme fuerzas en los momentos de debilidad y por alentarme a seguir a pesar de todo. A Segundo, Mia y Kalita, por su hermosa compañía día a día.}

\emph{Por último, pero no menos importante, a Gaby y Euge mis compañeras de CEFOBI, por acompañarme en las partes más empedradas del camino, por compartir las risas y las  lágrimas, por su amistad y consejos. No hubiese alcanzado mucho de mis logros sin su ayuda, compañía y aliento en todo momento.}
\end{spacing}



%% Por si alguien tiene curiosidad, este "simp'atico" agradecimiento est'a
%% tomado de la "Tesis de Lydia Chalmers" basada en el universo del programa
%% de televisi'on Buffy, la Cazadora de Vampiros.
%% http://www.buffy-cazavampiros.com/Spiketesis/tesis.inicio.htm

%% Los cap'itulos inician con \chapter{T'itulo}, estos aparecen numerados y
%% se incluyen en el 'indice general.
%%
%% Recuerda que aqu'i ya puedes escribir acentos como: 'a, 'e, 'i, etc.
%% La letra n con tilde es: 'n.


\chapter*{Abreviaturas y Símbolos}
%\thispagestyle{empty}
\begin{description}

\item{EMA}: ensayos multiambientales.
\item{IGA}: interacción genotipo ambiente.
\item{NCOI}: interacción sin cambio de rango, del inglés \emph{no crossover interaction}
\item{COI}:  interacción con cambio de rango, del inglés \emph{crossover interaction}
\item{ANOVA}: análisis de la variancia, del inglés \emph{analysis of variance}
\item{AMMI}: modelo de los efectos principales aditivos y interacción multiplicativa, del inglés \emph{Additive Main effects and Multiplicative Interaction}
\item{ACP}: análisis de componentes principales
\item{SREG}: modelo de regresión por sitio, del inlés \emph{Site Regression model}
\item{DVS}: descomposición de valores singulares
\item{GNU}: \emph{General Public Licence}
\item{CRAN}: \emph{Comprehensive R Archve Network}
\item{EM}: maximización de la esperanza, del inglés \emph{Expectation-Maximization}
\end{description}

%% Las secciones del "prefacio" inician con el comando \prefacesection{T'itulo}
%% Este tipo de secciones *no* van numeradas, pero s'i aparecen en el 'indice.
%%
%% Si quieres agregar una secci'on que no vaya n'umerada y que *tampoco*
%% aparesca en el 'indice, usa entonces el comando \chapter*{T'itulo}
%%
%% Recuerda que aqu'i ya puedes escribir acentos como: 'a, 'e, 'i, etc.
%% La letra n con tilde es: 'n.

\chapter*{Resumen}

%\thispagestyle{empty}

Los programas informáticos se han convertido, hoy en día, en una herramienta básica utilizada por el análisis estadístico como apoyo fundamental a la hora de realizar diferentes operaciones y para facilitar una mayor comodidad a los usuarios. Actualmente, R es uno de los programas más utilizados debido a su potencia y a su distribución como software libre. 



\textbf{Palabras Clave:}

%% Por si alguien tiene curiosidad, este "simp'atico" agradecimiento est'a
%% tomado de la "Tesis de Lydia Chalmers" basada en el universo del programa
%% de televisi'on Buffy, la Cazadora de Vampiros.
%% http://www.buffy-cazavampiros.com/Spiketesis/tesis.inicio.htm

\include{abstract}


%% # 'Indices y listas de contenido #
%% Quitar los comentarios en las lineas siguientes para obtener listas de
%% figuras y cuadros/tablas.
\setcounter{secnumdepth}{4}
\setcounter{tocdepth}{4}

\addtocontents{toc}{\hspace{-7.5mm} \textbf{Capítulos}}
\addtocontents{toc}{\hfill \textbf{Página} \par}
\addtocontents{toc}{\vspace{-2mm} \hspace{-7.5mm} \hrule \par}
\tableofcontents
\cleardoublepage

\listoffigures
\cleardoublepage

%%\renewcommand{\listtablename}{Índice de tablas}
%%\renewcommand{\tablename}{Tabla}
%%\listoftables
%%\cleardoublepage

%% # Cap'itulos #
%% Por cada cap'itulo hay que crear un nuevo archivo e incluirlo aqu'i.
%% Mirar el archivo "intro.tex" para un ejemplo y recomendaciones para
%% escribir.

%%Lista de comentarios
%%\pagestyle{plain}
%%\mylistoftodos
%%\cleardoublepage

\pagestyle{plain}
\pagenumbering{arabic}

%% Los cap'itulos inician con \chapter{T'itulo}, estos aparecen numerados y
%% se incluyen en el 'indice general.
%%
%% Recuerda que aqu'i ya puedes escribir acentos como: 'a, 'e, 'i, etc.
%% La letra n con tilde es: 'n.


\chapter{Introducción}

A lo largo de la historia de la agricultura, el hombre ha desarrollado el mejoramiento vegetal (o fitomejoramiento) en forma sistemática y lo ha convertido en un instrumento esencial para la mejora de la producción agrícola en términos de cantidad, calidad y diversidad.  

La humanidad depende, directa o indirectamente, de las plantas. Para la alimentación, ya que todos sus alimentos son vegetales o se derivan de éstos por ejemplo: carne, huevos y productos lácteos. De las plantas se deriva también la mayoría de las fibras textiles, fármacos, combustibles, lubricantes y materiales de construcción.

El fitomejoramiento, en un sentido amplio, es el arte y la ciencia de alterar o modificar la herencia de las plantas para obtener cultivares (variedades o híbridos) mejorados genéticamente, adaptados a condiciones específicas, de mayores rendimientos económicos y de mejor calidad que las variedades nativas o criollas \citep{Allard67}. En otras palabras, el fitomejoramiento busca crear plantas cuyo patrimonio hereditario esté de acuerdo con las condiciones, necesidades y recursos de los productores rurales, de la industria y de los consumidores, o sea de todos aquellos que producen, transforman y consumen productos vegetales. 

Las variedades mejoradas son el resultado del trabajo de desarrollo genético llevado a cabo en los programas de fitomejoramiento, los cuales se extienden a lo largo de varios años y requieren cuantiosas inversiones. La vigencia comercial de las variedades puede extenderse durante varias décadas, por lo que su elección es crítica para que el productor evite pérdidas económicas por malas campañas y el suministro al mercado sea constante. 

Generalmente, en etapas tempranas de estos programas existen un gran número de genotipos experimentales con pocos antecedentes de evaluación; mientras que en etapas posteriores  se trabaja con pocos genotipos altamente selectos. 

La utilización adecuada de procedimientos de análisis de datos agronómicos y ambientales es una condición inherente al desarrollo actual y futuro de investigaciones orientadas a mejorar los cultivos en forma económica y ambientalmente sustentable. En etapas avanzadas de los programas de mejoramiento, los ensayos multiambientales (EMA) que comprenden experimentos en múltiples ambientes son herramientas fundamentales para incrementar la productividad y rentabilidad de los cultivos. Estos son frecuentes en investigaciones agricolas de comparación de rendimiento, ya que constituyen una de las principales estrategias de identificación de genotipos vegetales superiores y de ambientes en los cuales estos se expresan de manera diferencial.

Debido a que las regiones de producción de los principales cultivos cubren áreas ecológicas muy extensas, se observan variaciones en las condiciones climáticas y de suelo. Por lo tanto, la aparición de la interacción genotipo ambiente (IGA) es inevitable, provocando respuestas altamente variables en los diferentes ambientes (Crossa et al., 1990; Cruz Medina, 1992; Kang y Magari, 1996). La IGA es considerada casi unánimemente por los fitomejoradores como el principal factor que limita la respuesta a la selección y, en general, la eficiencia de los programas de mejoramiento. 

Los investigadores agrícolas han sido conscientes de las diversas implicaciones de IGA en los programas de mejoramiento (Mooers, 1921; Yates y Cochran, 1938). Por ejemplo, la IGA tiene un impacto negativo en la heredabilidad, cuanto menor sea la heredabilidad de un caracter, mayor será la dificultad para mejorar ese caracter mediante la selección.
Por lo tanto, información sobre la estructura y la naturaleza de la IGA es particularmente útil para los mejoradores porque puede ayudar a determinar si necesitan desarrollar cultivares para todos los ambientes de interés o si deberían desarrollar cultivares específicos para ambientes específicos (Bridges, 1989). Gauch y Zobel (1996) explicaron la importancia de IGA como: “Si no hubiera interacción, una sola variedad de trigo (\emph{Triticum aestivum} L.) o maíz (\emph{Zea mays} L.) o cualquier otro cultivo rendiria al máximo en todo el mundo, y además la prueba de variedades deberia realizarse en un sólo lugar para proporcionar resultados universales. No habría ruido, los resultados experimentales serían exactos, identificando la mejor variedad sin error, y no habría necesidad de replicación. Entonces, una réplica en un lugar identificaría la mejor variedad de trigo que florece en todo el mundo ".

Peto (1982) ha distinguido las interacciones cuantitativas, llamada tambien sin cambio de rango (NCOI), o \emph{no crossover}, de las interacciones cualitativas, denominada tambien con
cambio de rango (COI) o \emph{crossover}(Cornelius et al., 1996). Cuando dos genotipos X e Y tienen una respuesta diferencial en dos ambientes diferentes, pero su ordenación permanece sin cambios se dice que la IGA es \emph{no crossover} (Figura \ref{fig:fig11}(A)). Sin embargo, es de tipo \emph{crossover} cuando hay cambios en el orden de los genotipos (Figura  \ref{fig:fig11}(B)). Cuando los genotipos responden de manera similar en ambos ambientes (Figura \ref{fig:fig11}(C)) no hay IGA. 


\begin{figure}[h]
\begin{center}
\includegraphics[width=14cm]{./Graficos/interac}
\end{center}
\caption{Representación gráfica de tipos de IGA: (A)IGA no crossover, (B) IGA crossover y (C) no IGA}
\label{fig:fig11}
\end{figure}


Distintos conceptos como regiones ecológicas, ecotipos, mega-ambientes, adaptaciones de germoplasma tanto en sentido amplio (a través de los ambientes) como específico (para cada ambiente o grupos de ambiente particular) (Kang et al., 2004) se pueden analizar a partir de la interacción genotipo-ambiente (Yan y Hunt, 2001).


Un análisis adecuado de la información de los EMA es indispensable para que el programa de mejoramiento de los cultivos sea eficaz. El rendimiento medio en los ambientes es un indicador suficiente del rendimiento genotípico solo en ausencia de IGA (Yan y Kang, 2003). Sin embargo, la aparición de IGA es inevitable y no basta con la comparación de las medias de los genotipos, sino que se debe recurrir a una metodología estadística más aporopiada. La metodología estadística más difundida para analizar los datos provenientes de EMA se basa en modificaciones de los modelos de regresión, análisis de variancia (\emph{Analysis of Variance}, ANOVA) y técnicas de análisis multivariado. 


Particularmente, para el estudio de la interacción y los análisis que de ella se derivan, dos modelos multiplicativos han aumentado su popularidad entre los fitomejoradores como una herramienta de análisis gráfico: el modelo de los efectos principales aditivos y interacción multiplicativa (\emph{Additive Main effects and Multiplicative Interaction}, AMMI) (Kempton 1984, Gauch, 1988), y el de regresión por sitio (\emph{Site Regression model}, SREG) (Cornelius et al., 1996; Crossa y Cornelius, 1997 y 2002).  Estos modelos combinan un ANOVA con la descomposición de valores singulares (DVS) o el análisis de componentes principales (ACP) sobre la matriz residual de ANOVA. En SREG, el ANOVA se realiza sobre el efecto principal de A mientras que en AMMI se considera el efecto de G y A. Mientras que a través del modelo AMMI se obtiene el gráfico biplot \emph{Genotipe-Environment} (GE) el cual es usado para explorar patrones puramente atribuibles a los efectos GE, para el modelo SREG, Yan y Hunt (2002), presentaron la técnica GGE biplot usado para explorar simultáneamente patrones de variación en la suma G+GE.

Una limitación importante de la mayoría de las propuestas de análisis provenientes de EMA es que requieren que el conjunto de datos este completo. Aunque los EMA están diseñados para que todos los genotipos se evalúen en todos los ambientes,  las tablas de datos genotipo x ambiente completas son poco frecuentes (no todos los genotipos se encuentran en todos los ambientes). Esto ocurre, por ejemplo, debido a errores de medición o causas naturales como, por ejemplo, la destrucción de plantas por animales, inundaciones o durante la cosecha, la incorporación de nuevos genotipos y a que otros se descartan por su pobre desempeño (Hill y Rosemberg, 1985). En estos casos, entre las posibles soluciones para tratar con una tabla de datos incompleta es (i) el uso de un subconjunto completo de datos, eliminando aquellos genotipos que tienen valores faltantes (Ceccarelli et al., 2007, Yan et al., 2011), (ii) completar datos faltantes con la media ambiental, o (iii) imputación de datos faltantes con valores estimados utilizando, por ejemplo, un modelo multiplicativo (Kumar et al., 2012). 

En este contexto, el análisis de datos provenientes de EMA requiere metodología estadística sofisticada cuyas rutinas informáticas se encuentran disponibles en programas desarrollados por diferentes empresas. Esto genera el inconveniente de tener que disponer de todos los programas necesarios para los distintos análisis, atender los requerimientos de formatos de datos usados por cada uno, y comprender los diversos tipos de salidas en las que se ofrecen los resultados obtenidos. Además, algunos procedimientos, especialmente aquellas metodologías recientes, no se encuentran dispobibles, y los costos de las licencias de dichos programas resultan muy elevados. 


El software R se trata de un proyecto de software libre distribuido bajo los términos de la \emph{General Public Licence} (GNU), desarrollado por \emph{The R Foundation for Statistical Computing}. Surge como resultado de la implementación de uno de los lenguajes más utilizados en investigación por la comunidad estadística, el lenguaje S. Que un software sea libre quiere decir que sus usuarios son libres de usarlo, copiarlo, distribuirlo, editarlo y modificarlo según sus propias inquietudes (FSF 2019). A diferencia de los programas estadísticos utilizados frecuentemente, R es un lenguaje de programación y no dispone de una interfaz gráfica en la cual se utilizan menues para realizar los distintos análisis de interes, lo cual genera dificultad en su uso para aquellos que no se encuentran familiarizados con el uso de código computacional. Sin embargo, brinda mayores posibilidades en cuanto a la manipulación y análisis de los datos ya que le permite a los usuarios definir sus propias funciones y pesonalizar el tipo de análisis que desean realizar. Si bien la versión básica del programa dista mucho de ser amigable, R Studio permite contar con una interacción más fluida con el programa R actuando como una interfaz amigable entre el usuario.  RStudio es un entorno de desarrollo integrado (IDE) gratuito y de código abierto para R. 

R forma parte de un proyecto colaborativo ya que promueve el hecho de que los usuario creen funciones y las ponga al alcance de toda la comunidad, es decir que está en continuo desarrollo y actualización.  Sin embargo, como muchas veces no resulta sencillo reutilizar una función creada por algun usuario se ha introducido la posibilidad de crear paquetes (\emph{package}) o librerías. Estas son una colección de objetos creados y organizados siguiendo un protocolo fijo que garantiza un soporte mínimo para el usuario así como la ausencia de errores (de sintaxis) en la programación.

R cuenta con 14 paquetes básicos y 29 recomendados para su funcionamiento instalados automaticamente en él, como por ejemplo, \emph{base} o \emph{stats}. Dado que la comunidad de usuarios que programan en R ha ido creciendo notablemente en los últimos años y que muchos de ellos han ido proporcionando librerías, se cuenta con una gran cantidad de paquetes que extienden las funciones básicas de R. Entre ellos se encuentran, \emph{plyr}, \emph{lubridate}, \emph{reshape2} y \emph{stringr} para la manipulación de los datos; \emph{ggplot2} y \emph{rgl} para la visualización; \emph{knitr} y \emph{xtable} para la presentación de resultados; entre otros. La lista completa de los paquetes oficiales puede consultarse en CRAN\footnote{CRAN (Comprehensive R Archve Network) es el repositorio oficial de paquetes de R, el lugar donde se publican las nuevas versiones del programa, etc. Contiene la lista completa de paquetes oficiales. \url{https://cran.r-project.org/web/packages/available_packages_by_name.html}}, se contaba con más de 14.000 paquetes disponibles en CRAN hasta junio de 2019. Esta gran variedad de paquetes es una de las razones por las que R tiene tanto éxito: lo más probable es que alguien ya haya resuelto un problema en el que estás trabajando, y puedes beneficiarte de su trabajo descargando su paquete.

Además de los paquetes oficiales, existen otros que pueden instalarse desde repositorios como, por ejemplo, Github. Sin embargo, no es sencillo encontrar un paquete que puede ser útil para un determinado fin sino que se debe recurrir a varios de ellos para cumplir un determinado objetivo. 

La decisión de qué software de análisis estadístico utilizar no tiene una respuesta predeterminada: la elección dependerá de las necesidades de la investigación. Esto pues los lenguajes de programación son herramientas y el principal criterio para decidir el uso de uno u otro debe efectuarse en función de la particularidad de los objetivos y alcances de la investigación que se busque desarrollar.

Frecuentemente, los mejoradores utilizan programas que tienen una interfaz gráfica para realizar los análisis estadísticos deseados y no tienen un manejo fluido de un lenguaje de programación. En el año 2012 se creó el paquete \emph{Shiny} de R que permite desarrollar aplicaciones Web utilizando R, acercando la potencia de R a todo tipo de usuarios.


El objetivo del presente trabajo fue: (i) crear un paquete de R que incluya las funciones que permitan analizar los datos provenientes de EMA, incluyendo además metodología recientemente publicada que no se encuentra disponible en R; (ii) crear una interfaz gráfica, entre R y el usuario, mediante Shiny con el fin de poder realizar los análisis disponibles en el paquete creado sin necesidad de utilizar el lenguaje de programación.



Elousa, Paula. 2009. “¿EXISTE VIDA MÁS ALLÁ DEL SPSS? DESCUBRE R.” Revista Psicothema 21 (4): 652–55. http://www.psicothema.com/psicothema.asp?id=3686.

FSF. 2019. “¿Qué Es El Software Libre?” Free Software Foundation. https://www.fsf.org/es/recursos/que-es-el-software-libre.
\chapter{Objetivos}

\section{Objetivo general}
 
Desarrollar un paquete de R para el análisis de datos provenientes de EMA y una interfaz gráfica de usuario para el mismo a través de la aplicación web Shiny.


\section{Objetivos específicos}
\begin{itemize}
\item Mostrar un flujo de trabajo reproducible para la construcción de paquetes de R.
\item Programar e incluir en el paquete metodología para el análisis de datos provenientes de EMA recientemente publicada y no disponible en R.
\item Añadir en el paquete de R funciones ya existentes con modificaciones o agregados para favorecer su uso.
\item Desarrollar una aplicación web Shiny que sirva como interfaz gráfica de usuario para el paquete.
\item Publicar el paquete y la aplicación web para su libre uso.
\end{itemize}


%% Los cap'itulos inician con \chapter{T'itulo}, estos aparecen numerados y
%% se incluyen en el 'indice general.
%%
%% Recuerda que aqu'i ya puedes escribir acentos como: 'a, 'e, 'i, etc.
%% La letra n con tilde es: 'n.

\chapter{Métodos}
%\setcounter{section}{1}
\section{Métodos estadísticos}

\subsection{Modelo AMMI y SREG}

El modelo AMMI propuesto por Zobel et al. (1988) es un modelo multiplicativo en el cual se expresa el fenotipo de un genotipo en un ambiente de la siguiente forma:
\begin{center}
$y_{ij}= \mu +G_i + A_j + \sum_{k=1}^q \lambda_k \alpha_{ik} \gamma_{jk}$ \hspace{0.5cm} $ i=1,...,g$;\hspace{0.15cm} $ j=1,...,a$;\hspace{0.15cm} $q=min(g-1,a-1)$
\end{center} 
donde 
\begin{itemize}
\item $y_{ij}$ es el caracter fenotípico evaluado (rendimiento o cualquier otro caracter de interes) del $i$-ésimo genotipo en el $j$-ésimo ambiente,
\item $\mu$ es la media general,
\item  $G_i$ es el efecto del $i$-ésimo genotipo,
\item $A_j$ es el efecto del $j$-ésimo ambiente
\item $\sum_{k=1}^q \lambda_k \alpha_{ik} \gamma_{jk}$ es la sumatoria de componentes multiplicativas utilizadas para modelar la IGA. Siendo, $\lambda_k$ el valor singular para la  $k$-ésima componente principal (PC) $\alpha_{ik}$ y $\gamma_{jk}$ son los scores de las PC para el $i$-ésimo genotipo y el $j$-ésimo ambiente para la $k$-ésima componente, respectivamente.
\end{itemize}

En cambio, el modelo SREG (Cornelius et al., 1996; Crossa y Cornelius, 1997 y 2002) expresa el fenotipo observado en función del efecto ambiental en forma aditiva y del genotipo e interacción agrupados y en forma multiplicativa:
\begin{center}
$y_{ij}= \mu +  A_j + \sum_{k=1}^q \lambda_k \alpha_{ik} \gamma_{jk}$ \hspace{0.5cm} $ i=1,...,g$;\hspace{0.15cm} $ j=1,...,a$; \hspace{0.15cm} $q=min(g-1,a)$
\end{center} 

Los parámetros multiplicativos, tanto en el modelo AMMI como en el SREG, se estiman por medio de la Descomposición en Valores Singulares (DVS) de la matriz que contiene los residuos del modelo aditivo luego de ajustar por mínimos cuadrados el modelo de efectos principales. Generalmente los dos primeros términos multiplicativos son suficientes para explicar los patrones de la IGA, así como de G e IGA en forma conjunta; la variabilidad remanente se interpreta como ruido. 
 
 Gabriel (1971) presentó el concepto del biplot que consiste en la representación de las filas (individuos) y las columnas (variables) de una matriz de datos en un mismo gráfico. Éstos biplots, son herramientas poderosas para el análisis e interpretación de la estructura de datos provenientes de ensayos multiambientales utilizados en los programas de mejoramiento (Ebdon y Gauch, 2002; Samonte et al., 2004; Yan et al., 2000; Zobel et al., 1988). Del modelo AMMI se obtiene el biplot GE (\emph{Genotipe-Environment}) (\textbf{CITA})  que permite interpretar la variación producida por los efectos de la IGA; mientras que, en el biplot GGE (\emph{Genotipe} plus \emph{Genotipe-Environment}) (Yan et al., 2000), derivado del modelo SREG, se analizan conjuntamente el efecto de G e IGA.

Dado que para seleccionar cultivares, el efecto de G e IGA debe considerarse simultáneamente, el modelo SREG resulta superior a AMMI para visualizar patrones en datos EMA. El biplot GGE permite investigar la existencia de megaambientes (grupo de ambientes en donde los cultivares de mejor desempeño son los mismos) entre los ambientes en estudio, seleccionar cultivares superiores en un megaambiente dado y seleccionar los mejores ambientes de evaluación para analizar las causas de la IGA.

\textbf{Hablar un poco de los metodos de SVD del modelo SREG que da lugar a distintos graficos. Un oracion tipo dependiendo del escalado utilizado se pueden obtener distintas interpretaciones.... o no... no se... pero despues en resultados presentamos distintos escalados ... aunq pongo que vayan a leeer a tal autor nose... }

\subsection{Modelo AMMI robusto}

El modelo AMMI, en su forma estándar, asume que no hay valores atípicos en el conjunto de datos. La presencia de \emph{outliers} es más una regla que una excepción cuando se consideran datos agronómicos debido a errores de medición, algunas plagas / enfermedad que puede influir en algunos genotipos  resultando por ejemplo en un rendimiento inferior al esperado en un ambiente, o incluso debido a alguna característica inherente de los genotipos que se evalúan.

Rodrigues et al. (2015) proponen una generalización robusta del modelo AMMI, que resulta de ajustar la regresión robusta basada en el estimador M-Huber (Huber, 1981) y luego utilizar un procedimiento DVS / PCA robusto. Consideraron varios métodos de DVS / PCA dando lugar a un total de cinco modelos robustos llamados: R-AMMI, H-AMMI, G-AMMI, L-AMMI, PP-AMMI. 

El empleo de la versión robusta del modelo AMMI puede ser extremadamente útil debido a que una mala representación de genotipos y ambientes puede resultar en un mala decisión con respecto a qué genotipos seleccionar para un conjunto dado de ambientes (Gauch1997,Yanetal2000). A su vez, la elección de los genotipos incorrectos pueden provocar grandes pérdidas en términos de rendimiento. Los biplots obtenidos de los modelos robustos mantienen las características e interpretación estándar del modelo AMMI clásico (Rodrigues et al., 2015).


\subsection{Métodos de imputación}


Una limitación importante que presentan los modelos multiplicativos descriptos previamente es que requieren que el fenotipo de todas las combinaciones de genotipos y ambientes se encuentre registrado, es decir no admiten valores perdidos. Aunque los EMA están diseñados para que todos los genotipos se evalúen en todos los ambientes, la presencia de valores faltantes es muy común debido a errores de medición o pérdidas de plantas por animales, inundaciones o problemas durante la cosecha, además de la dinámica propia de la evaluaciones en las que se incorporan y se descartan genotipos debido a su pobre desempeño (Hill y Rosemberg, 1985)

Se han propuesto numerosas metodologías para superar el problema de valores ausentes en el conjunto de datos, entre las cuales se encuentran:

\begin{itemize}
\item EM-AMMI: Gauch y Zobel (1990) desarrollaron un procedimiento iterativo que utiliza el algoritmo de maximización de la esperanza (EM, del inglés \emph{Expectation-Maximization}) incorporando el modelo AMMI. 
\end{itemize}
\begin{itemize}
\item EM-SVD: Perry (2009a) propone un método de imputación que combina el algoritmo EM con DVS. 
\end{itemize}
\begin{itemize}
\item EM-PCA: Josse y Husson (2013) proponen imputar los valores faltantes de un conjunto de datos con el modelo de Análisis de componentes principales.
\end{itemize}
\begin{itemize}
\item Gabriel Eigen: Arciniegas-Alarcón et al. (2010) propuso un método de imputación que combina regresión y aproximación de rango inferior usando DVS. 
\end{itemize}
\begin{itemize}
\item WGabriel Eigen: 
\end{itemize}




\section{Paquete de R}

%https://oscarperpinan.github.io/R/Paquetes.html 


Un paquete de R es la unidad básica para la distribución de código de R, que reúne de forma estructurada funciones, datos, archivos con documentación y testeos.  Para la creación del mismo se deben seguir ciertas convenciones, existiendo elementos obligatorios y otros opcionales, entre los primeros se encuentran:

\begin{itemize}
	\item Archivo DESCRIPTION: describe el contenido del paquete y establece cómo se va a relacionar con otros.
\end{itemize}

\begin{itemize}
	\item Carpeta R: contiene el o los archivos de código de R con las funciones del paquete.
\end{itemize}

\begin{itemize}
	\item Carpeta man: incluye archivos con la documentación del paquete, funciones y datasets.
\end{itemize}

\begin{itemize}
	\item Archivo NAMESPACE: declara las funciones del paquete que se ponen a disposición de los usuarios y de qué funciones de otros paquetes hace uso.
\end{itemize}


Los elementos opcionales que se pueden agregar son por ejemplo:

\begin{itemize}
	\item Carpeta data: contiene objetos de R que contienen datos.
\end{itemize}

\begin{itemize}
	\item Carpeta vignettes: contiene los tutoriales que muestran ejemplos de uso del paquete, generalmente escritos en Rmarkdown.
\end{itemize}

\begin{itemize}
	\item Carpeta tests: incluye código que permiten someter al paquete a diversos controles.
\end{itemize}



Para la creación del paquete, se deben instalar y cargar en la sesión de trabajo los paquetes:
 \emph{devtools}, \emph{usethis}, \emph{roxygen2}, \emph{ustestthat}, \emph{knitr}, \emph{available}. Además, en caso de utilizar el Windows se debe descargar e instalar Rtools .


\subsection{Creación del paquete}

En primer lugar se debe elegir el nombre del paquete, el cual debe cumplir con ciertas reglas: solo puede contener letras, números o puntos; tener al menos dos caracteres y empezar con una letra y no terminar con un punto. Una vez elegido el nombre, se debe chequear si el mismo está disponible en los repositorios \emph{GitHub}, \emph{CRAN} y \emph{Bioconductor}, donde se alojan los paquetes. Para ello, se utiliza el paquete \emph{available}, que además indicará si el nombre elegido tiene algún significado especial que podemos desconocer (revisa las webs de Wikipedia, Wiktionary y Urban Dictionary) (Figura \ref{fig:fig31}). 

\begin{lstlisting}
	# Cargar la libreria devtools
	library(available)
	# Crear el paquete geneticae
	available(``geneticae")
\end{lstlisting}


 \begin{figure}[H]
\begin{center}
	\includegraphics[width=9.5cm]{./Graficos/available.png}	
\end{center}
	\caption{Chequeo de disponibilidad del nombre elegido}
\label{fig:fig31}
\end{figure}


Para la creación del paquete se utilizán  \emph{devtools} y  \emph{usethis} que incluyen funciones que simplifican la tarea. La función \textcolor{blue}{create\_package}("nombre\_paquete") generará una carpeta con el nombre provisto (Figura \ref{fig:fig32}). Si nose especifica una ubicación entonces se creará en el directorio actual.


\begin{lstlisting}
# Cargar la libreria devtools
library(devtools)
# Crear el paquete geneticae
create_package(``/home/julia-fedora/Escritorio/geneticae")
\end{lstlisting}


 \begin{figure}[H]
	\begin{center}
		\includegraphics[width=13cm]{./Graficos/creacion.png}	
	\end{center}
	\caption{Creación del paquete geneticae}
	\label{fig:fig32}
\end{figure}


\subsection{Archivos de código}

Una vez creada la estructura del paquete se deben programar las funciones que el mismo contendrá. Cada una de ellas debe ser guardada en un archivo de extensión .R, en el subdirectorio R/. Para ello, se utiliza la función \textcolor{blue}{use\_r}() la cual crea un script ubicado en la carpeta R/, donde el código de interés será agregado. 

A medida que se va desarrollando el paquete, con funciones internas y otras que se exportan, con algunas que se relacionan entre si y que a su vez dependen de otros paquetes, se deben ir realizando pruebas para asegurarse que los creados códigos realizan lo que realmente se desea. Para ello, la función \textcolor{blue}{load\_all}() simula el proceso de construcción, instalación y carga del paquete, permitiendo probar la función de manera interactiva.

Muy frecuentemente se utilizan funciones que se encuentran disponibles en otros paquetes. La función \textcolor{blue}{use\_package}() agrega el paquete de interés a la sección Imports del archivo DESCRIPTION, y luego para llamar a las mismas dentro de una función se debe utilizar @importFrom paquete función. Alternativamente, si se utilizan repetidamente muchas funciones de otro paquete, es posible importarlas todas utilizando @import paquete. Sin embargo, esta es la solución menos recomendada porque hace que el código sea más difícil de leer, y si tiene muchos paquetes, aumenta la posibilidad de que entren en conflicto nombres de funciones.


\subsection{Documentación}

Uno de los aspectos más importantes del paquete es la documentación donde se describe cómo se usa cada función, para qué sirven los argumentos, aclarar qué tipo de resultado devuelve, proveer ejemplos para el uso, etc. El paquete \emph{roxygen2}, provee pautas para escribir comentarios con un formato especial que incluyan toda la información requerida justo antes de la definición de la función. El código y la documentación son adyacentes, de modo que cuando el código se modifique le exigirá que actualice la documentación. 

El flujo de trabajo para crear la documentación con el paquete \emph{roxygen2} es el siguiente:

\begin{itemize}
\item Agregar comentarios a los archivos .R. Estos deben comenzar con \#', para distinguirlo de los comentarios regulares, y preceden a una función. La primera oración se convierte en el título y el segundo párrafo es una descripción de la función. Para el resto de los campos de la documentación, se utilizan de etiquetas que comienzan con @, siendo las más importantes a incluir:

\begin{itemize}
\item @param: se detalla para qué sirve cada parámetro de la función.
\item @return: para explicar qué objeto devuelve la función.
\item @details: agregar cualquier aclaración que se considere necesaria.
\item @examples: incluir ejemplos de uso de la función.
\item @export: indicar que esta función tiene que estar disponible cuando alguien cargue el paquete con library(). No es necesario exportar funciones auxiliares de utilidad interna.
\end{itemize}

\item Ejecutar devtools::document() para convertir los comentarios de roxygen en archivos .Rd que compondrán el manual y que deben ir guardados en la carpeta man.
\end{itemize}


\emph{Roxygen2} permite utilizar la descripción de los parámetros de otras funciones usando @inheritParams. Esta documentará los parámetros que no están documentados en la función actual, pero que si lo están en la función fuente. La fuente puede ser una función en el paquete actual, vía @inheritParams function, u otro paquete, vía @inheritParams package::function. Además \emph{Roxygen2} permite incluir referencias utilizando @references. En caso de importar paquetes, como se indicó en la sección anterior, se deben declarar usando @importFrom o @import, previo a la definición de la función.


\subsection{Editar el archivo DESCRIPTION}

El archivo DESCRIPTION provee toda la metadata sobre el paquete que se esta creando. En este archivo hay algunos campos que tienen que estar presentes de forma obligatoria y otros que son opcionales. Los elementos obligatorios son:

\begin{itemize}
\item Package: nombre del paquete
\item Title: título del paquete (hasta 65 caracteres, Escrito De Esta Forma).
\item Version: número de la versión actual del paquete (por ejemplo, 0.2.1)
\item Author, Maintainer o Authors@R: quiénes han participado en el paquete.
\item Description: un párrafo que describa el paquete.
\item License: nombre de la licencia bajo la cual se distribuye el paquete. Si se pretende que cualquiera lo puede usar, entonces se debe recurrir a los tipos mas comunes de licencia para código abierto: CC0, MIT o GPL. Como se muestra en la Figura \ref{fig:fig33}, el paquete geneticae se encuentra bajo la licencia GPL-3. Para esto, se utilizó la función \textcolor{blue}{use\_gpl3\_license}() del paquete usethis, la cual agrega la información al archivo DESCRIPTION y además crea un archivo LICENSE al directorio del paquete.
\end{itemize}




En cambio, los elementos no obligatorios:
\begin{itemize}
\item Date: fecha de publicación de esta versión del paquete.
\item Imports, Depends, Suggests: es muy común que las funciones desarrolladas necesiten hacer uso de algunas que pertenecen a otros paquetes. Estos serán indicados en los campos Imports, Depends, Suggests del archivo DESCRIPTION. Como se muestra en la Figura \ref{fig:fig33}, en el campo Imports se indica que geneticae necesita los paquetes: stats, GGEBiplots, ggforce, ggplot2, etc. Mientras que los listados en Suggest indica que se podría hacer uso de los mismos, aunque no son indispensables. Por último, el paquete geneticae se puede utilizar en versiones de R iguales o superiores a la 2.12, como se establece en Depends.
\item URL: dirección de la página web del paquete.
\item BugReports: dirección donde los usuarios pueden enviar avisos con los problemas que encuentren al utilizar el paquete.
\end{itemize}

El archivo DESCRIPTION del paquete geneticae se muestra en la Figura \ref{fig:fig33}

 \begin{figure}[H]
	\begin{center}
		\includegraphics[width=13cm]{./Graficos/DESCRIPTION.png}	
	\end{center}
	\caption{Archivo DESCRIPTION de geneticae}
	\label{fig:fig33}
\end{figure}


\subsection{Testeos}

Las pruebas resultan fundamentales en el desarrollo de paquetes, asegura que el código haga lo que realmente se desea. Existen pruebas informales como aquellas realizadas con la función \textcolor{blue}{load\_all}() que permite que las funciones creadas estén disponible rápidamente para uso interactivo. Sin embargo, las pruebas interactivas pueden convertirse en scripts reproducibles, los cuales resultan superiores debido a que se indica explícitamente cómo debería comportarse el código, provocando que los errores solucionados no vuelvan a ocurrir. Para ello, se utiliza la función \textcolor{blue}{use\_testthat}() del paquete \emph{testthat} (Wickham,2011). Esta agrega testthat al campo Suggests en el archivo DESCRIPTION, crea un directorio tests/ para alojar cualquier tipo de unidad de testeo, una subcarpeta testthat donde se ubicaran los testeos escritos bajo este sistema y además, crea un archivo testthat.R, que se encarga de la ejecución de todos los testeos.  

Los testeos se organizan en tres niveles:
\begin{itemize}
\item Archivo de tests: uno por cada archivo .R en la carpeta R/.
\item Ejecutar pruebas automáticamente cada vez que algo cambie con la función \textcolor{blue}{autotest}(). Estas son útiles cuando las pruebas se ejecutan con frecuencia. Si se modifica un archivo de prueba, probará ese archivo; si se modifica un archivo de código, volverá a cargar ese archivo y volverá a ejecutar todas las pruebas.
\item Expectation: es el nivel más desagregado, corre cierto código y se compara el resultado obtenido con el esperado.
\end{itemize}

La función \textcolor{blue}{use\_test}(), creará los archivos de prueba cuyo nombre tienen que ser test-nombre\_archivo\_de\_codigo.R  y los ubicará en la carpeta test/testthat. Una vez escritos estos archivos, podemos evaluar los resultados de los testeos con \textcolor{blue}{devtools::test()}. Ante cada error encontrado, nos detenemos para corregirlo y repetimos este proceso hasta que todas las unidades de testeo pasen la prueba. En la Figura \ref{fig:fig34} se muestra el resultado de correr los test creados para el paquete geneticae.


 \begin{figure}[H]
	\begin{center}
		\includegraphics[width=11cm]{./Graficos/Test.png}	
	\end{center}
	\caption{Resultado de correr los tests creados para geneticae}
	\label{fig:fig34}
\end{figure}

Una medida de la calidad de un paquete está dada por el porcentaje de su código que es evaluado durante los testeos. El paquete covr permite hacer ese cálculo, además de mostrar interactivamente qué partes del código fueron evaluadas y cuáles no. Por un lado puede evaluarse en cada archivo .R mediante \textcolor{blue}{devtools::test\_coverage\_file()}, o bien, la cobertura total usando \textcolor{blue}{devtools::test\_coverage()}. El paquete geneticae tiene un porcentaje total de cobertura de los test igual a 16.83\% (Figura \ref{fig:fig35}).


 \begin{figure}[H]
	\begin{center}
		\includegraphics[width=14cm]{./Graficos/Cobertura.png}	
	\end{center}
	\caption{Análisis de cobertura de los tests de geneticae}
	\label{fig:fig35}
\end{figure}

{\LARGE{\textbf{ampliar los test para tener mas cobertura}}}



\subsection{Compilación e instalación}

La función \textcolor{blue}{check}() o R CMD check ejecutado en el shell, es utilizado para verificar que un paquete R esta en pleno funcionamiento. La misma verificará que no haya errores de sintaxis o no se generen warnings. Está compuesto por más de 50 chequeos individuales entre los cuales se encuentran: la estructura del paquete, el archivo descripción, namespace, el código de R, los datos, la documentación, entre otros.  La diferencia con los testeos realizados mediante \textcolor{blue}{use\_testthat}(), es que la última evalúa si las funciones desarrolladan realizan lo deseado, resultando propios de cada paquete. En cambio, lo realizado por R CMD check es común para todos los paquetes. 

Se aconseja realizar verificaciones completas de que todo funciona a medida que se van incorporando funciones ya que si se incorporan muchas y luego se verifican será difícil identificar y resolver los problemas. Una vez que se desarrollaron todos los elementos necesarios para el paquete y no se detectan errores, advertencias o notas, se ejecuta la función \textcolor{blue}{install}(), con el objetivo de instalar el paquete en la biblioteca.


\subsection{Algunos elementos complementarios}

Existen algunas componentes que no son obligatorias a la hora de desarrollar un paquete. 

\subsubsection{Viñetas}

Una viñeta es un tipo especial de documentación que puede agregarse al paquete para dar más detalles y ejemplos sobre el uso del mismo. En ella se brinda es una descripción el problema que el paquete está diseñado para resolver y muestra al lector cómo resolverlo. Se diferencian de las páginas de ayuda en que su adición es opcional y no sigue una estructura fija, dándole la libertad al autor de enseñar de la forma que más le guste cómo usar su paquete.

Muchos de los paquetes existentes tienen viñetas la cuales se pueden encontrar utilizando la función \textcolor{blue}{browseVignettes}(``packagename") si el mismo se encuentra instalado, sino deben consultarse en su página de CRAN, por ejemplo para el paquete \emph{dplyr}: \url{http://cran.r-project.org/web/packages/dplyr}. Cada viñeta proporciona el archivo fuente original, una página HTML o PDF y un archivo de código R. 

Las Viñetas se pueden construir de diversas formas, en este trabajo se utiliza se utiliza \textcolor{blue}{usethis::use\_vignette}(``tutorial\_del\_paquete''). La misma crea un directorio vignettes/, agrega las dependencias necesarias a DESCRIPTION y crea el archivo para redactar la viñeta. 


\subsubsection{Agregar datasets}

A menudo es útil incluir datos en un paquete a fin de proporcionar ejemplos de las funciones incluidas en él. Esto es posible realizarlo con la función \textcolor{blue}{usethis::use\_data}() que crea un archivo .RData y lo almacena en el directorio data/.  Notar que el archivo DESCRIPTION contiene el campo LazyData: true, lo cual genera que los conjuntos de datos no ocupen memoria hasta que sean usados.

Los objetos en la carpeta data siempre se exportan, por lo cual hay que agregar documentación para los mismos. A diferencia de las funciones que son documentadas directamente, para los objetos en data/, se debe crear un archivo y guardarlo en el directorio R/. Esto se puede hacer con roxygen en cualquier Rscript de la carpeta R, aunque se acostumbra juntar toda la documentación para todos los datasets en un único archivo llamado data.R.\\

\subsubsection{Archivo README}

Un archivo README es una forma de documentación de software que contiene información acerca de otros archivos en una carpeta. Usualmente es un archivo de texto plano que permite describir brevemente por qué y para qué alguien tendría que usar el paquete, a la vez que indicar cómo conseguirlo o instalarlo. Se diferencian de la viñeta ya que sólo presenta una descripción breve del paquete.


Para generar el README con R Markdown se utiliza la función \textcolor{blue}{use\_readme\_rmd}() la cual crea un archivo de Rmarkdown con una plantilla donde se escribirá el mismo y será además agregado a .Rbuildignore. Luego, al compilarlo con knitr se obtendrá un archivo README.md, que será la cara visible del paquete si, por ejemplo, en GitHub.

\textbf{badges y logo}

Las insignias o badges son unos íconos que señalan distintas características del paquete, como su nivel de maduración, el nivel de cobertura en el testeo, cantidad de descargas, número de versión, resultado de los controles de CRAN, etc.
Son visualmente muy atractivas y se colocan el en archivo README.     El paquete usethis trae un conjunto de funciones que generan automáticamente el código a incluir en el README.Rmd para agregar los badges: \textcolor{blue}{use\_badge}(badge\_name, href, src), \textcolor{blue}{use\_cran\_badge}(), \textcolor{blue}{use\_bioc\_badge}(), \textcolor{blue}{use\_lifecycle\_badge}(stage), 
\textcolor{blue}{use\_binder\_badge}(sturlpath = NULL) .


Muchos de los paquetes disponibles disponen de un logo con forma hexagonal, que generalmente termina en forma de sticker: los hexStickers. Estos permiten terminar de darle identidad a tu paquete y hacerlo más vistoso. Para crearlos existe el paquete HexSticker. Una vez creado el logo, le podemos pasar su ubicación a la función use\_logo(), que se encargará de darle el tamaño adecuado, guardarlo en la carpeta man del paquete y producir el código de Markdown para incluirlo en el README. 
La Figura \ref{fig:fig36} presenta un fragmento GitHub del paquete geneticae, donde se muestra parte del contenido del archivo README, badges y logo.

\begin{figure}[H]
	\begin{center}
		\includegraphics[width=11cm]{./Graficos/badges.png}	
	\end{center}
	\caption{Fragmento de README de geneticae}
	\label{fig:fig36}
\end{figure}




\subsubsection{Archivo NEWS}


Mientras que el README apunta a ser leído por nuevos usuarios, el archivo NEWS es para aquellos que ya usan el paquete.
Este archivo se encarga de contar los cambios presentes en cada versión nueva del paquete que publicamos.
Se sugiere usar Markdown para escribir este archivo y colocar un título principal para cada versión, seguido por títulos secundarios que describen lo realizado (cambios principales, bugs arreglados, etc.).
Si se trata de cambios impulsados por otras personas, por ejemplo, a través de sugerencias hechas en GitHub, se los menciona.
Una buena práctica es ir escribiendo este archivo cada vez que se realiza algo nuevo en el paquete.
La función que nos permite crear este archivo automáticamente es
\textcolor{blue}{usethis::use\_news\_md}(). Hasta el momento, del paquete geneticae solamente se cuenta con la versión de desarrollo (Figura \ref{fig:fig37}) 



\begin{figure}[H]
	\begin{center}
		\includegraphics[width=11cm]{./Graficos/News.png}	
	\end{center}
	\caption{Archivo NEWS de geneticae}
	\label{fig:fig37}
\end{figure}





\subsubsection{Crear una página web}

Para mayor visualización del paquete es posible crear una página web \footnote{Para visitar la página web de geneticae debe dirigirse a \url{https://...........}}. El paquete pkgdown está diseñado para la creación de un sitio web de manera rápida y sencilla. Utiliza todo lo creado hasta el momento y lo convierte automáticamente en una pagina web mediante la función \textcolor{blue}{pkgdown::build\_site()}.

 {\LARGE{\textbf{No aparecen los badges en la pag web}}}


\subsubsection{Publicación}


Por último, para que otros usuarios puedan utilizar el paquete es necesario compartirlo en alguno de los repositorios más populares de paquetes de R, entre los cuales se encuentran: CRAN, Bioconductor, GitHub, rOpenSci, R-Forge y RForge. El paquete \emph{geneticae} se encuentra en GitHub, por lo tanto, para instalar el mismo se debe utilizar \textcolor{blue}{devtools::install\_github(``jangelini/geneticae")}.


\section{Shiny APP}
Shiny es un paquete R que permite construir aplicaciones web directamente desde RStudio sin necesidad de conocer en profundidad los lenguajes HTML / CSS / JavaScript . Estas aplicaciones constituyen una interfaz gráfica entre el usuario y R, que permiten realizar un análisis a través de un navegador web sin necesidad de programar.

Una característica importante de las aplicaciones web creadas mediante Shiny es que son dinámicas e interactivas. Para que shiny funcione correctamente, es necesario tener instalado R 3.0.2 o cualquier versión posterior.

\subsection{Estructura de Shiny APP}

Las aplicaciones están compuestas por la interfaz de usuario, ui (\emph{user interfaz}), sección server y la función shinyApp(). 


\textbf{Interfaz}

La interfaz del usuario (user interface o ui, por sus siglas en inglés) controla el diseño de la aplicación, recibe los inputs y muestra los outputs en el navegador. En general, definir las características de la interfaz puede no resultar tan sencillo ya que muchas de sus herramientas están vinculadas a otros lenguajes de programación, por ejemplo HTML, CSS o JavaScript. Sin embargo, las funciones del paquete shiny facilitan la tarea sin necesidad de conocer en profundidad estos lenguajes.

\textbf{Server}

En la sección server se escribe el código de R que le indica a la app qué debe hacer y cómo debe funcionar, incluyendo la lectura y manipulación de datos, el armado de gráficos, el ajuste de modelos, etc. Para esto, se define una función que debe tener dos argumentos: input y output. Los mismos son listas que almacenan elementos de entrada (datos u opciones elegidas por el usuario a través de la ui) y elementos de salida para mostrar en la app (resultados, tablas, gráficos, mapas, etc.), respectivamente.


\textbf{Ejecución}

Por último, se llama a la función shinyApp(), cuyos dos argumentos principales son ui y server, es decir, cada uno de los elementos definidos anteriormente. Ejecutar esta función da como resultado el lanzamiento de la aplicación, la cual podremos utilizar dentro de RStudio o usando nuestro navegador (Google Chrome, Mozilla Firefox, Microsoft Edge, etc.). Es importante destacar que, al seguir estos pasos, la aplicación sólo funcionará mientras la sesión de RStudio desde la cual se lanzó siga vigente.


\subsection{Desarrollo de Shiny APP}

Una forma de desarrollar una  aplicación es a partir de un nuevo directorio con un sólo archivo llamado app.R, como se muestra a continuación. 

\begin{lstlisting}
library(shiny)
ui<- ...
server<- ...
shinyApp(ui = ui, server = server)
\end{lstlisting}

En este archivo se carga el paquete shiny, se define la interfaz de usuario, la función server y por último, se ejecuta función que permite construir e iniciar una aplicación. Al ejecutar la aplicación la misma aparecerá, de manera predeterminada, en una ventana emergente. Sin embargo, otras dos opciones se pueden configurar desde el menú desplegable de \emph{Run App}. Una de ellas es la ejecución en el panel del visor que permite verla al mismo tiempo que ejecuta el código. La segunda opción es ejecutar en un navegador externo mostrando la aplicación como la mayoría de los usuarios la verán. Dado que la sesión de R estará monitoreando la aplicación y ejecutando las ordenes dadas por el usuario, no se podrá ejecutar ningún comando.

En cualquier lenguaje de programación tener el código duplicado genera un desperdicio computacional y, lo que es más importante, aumenta la dificultad de mantener o depurar el código. Cuando se programa en R, se utilizan dos técnicas para lidiar con el código duplicado: guardar un valor usando una variable o utilizar una función para almacenar un cálculo. Ninguno de estos enfoques son apropiados en una Shiny APP, sino que se utilizan expresiones reactivas. Una expresión reactiva tiene una diferencia importante con una variable: sólo se ejecuta la primera vez que se llama y luego almacena en caché el resultado de la misma hasta que necesite actualizarse. La programación reactiva es un estilo de programación que enfatiza valores que cambian con el tiempo, y cálculos y acciones que dependen de esos valores. Esto es importante para las aplicaciones Shiny porque son interactivas: los usuarios cambian los inputs, lo que hace que la lógica se ejecute en el servidor que finalmente resultan en actualización de los outputs/resultados.

Entre los problemas que pueden surgir al crear una Shiny app se encuentran los errores inesperados, no se obtiene ningún error pero el valor obtenido es incorrecto, o bien todos los resultados son correctos, pero no se actualizan cuando se deben. Una vez localizada la fuente del error, la herramienta más poderosa es el depurador interactivo, éste detiene la ejecución y brinda una consola interactiva donde puede se ejecutar cualquier código para descubrir el error. Para iniciar el mismo, se puede agregar la función browser() en el código fuente, o bien agregar un punto de interrupción RStudio haciendo clic a la izquierda del número de línea.

Al modificar la aplicación, se la ejecuta para poder ver los cambios realizados, por lo tanto resulta esencial reducir la velocidad de iteración. La primera forma acelerar el proceso consiste en escribir el código, utilizar el atajo del teclado Cmd/Ctrl+ Shift+ Enter en lugar del botón ``Ejecutar aplicación'', experimentar interactivamente con la aplicación y cerrar la aplicación, repitiendo este proceso al realizar cualquier cambio. Otra forma de reducir aún más la velocidad de iteración es activar la recarga automática (options(shiny.autoreload = TRUE)) y luego ejecutar la aplicación en un trabajo en segundo plano. Con este flujo de trabajo cuando se guarde un archivo, su aplicación se reiniciará: no es necesario cerrarla y reiniciarla, lo cual conduce a un flujo de trabajo aún más rápido. La principal desventaja de esta técnica es que debido a que la aplicación se ejecuta en un proceso separado, es considerablemente más difícil de depurar.


\subsection{Compartiendo una Shiny Web App}

Una vez creada la aplicación se la publica para su libre uso. En este caso la Shiny Web App encuentra disponible en el servidor de CONICET \url{www.cefobi.com}. Además el proyecto se encuentra en GitHub \url{https://github.com/jangelini/shinyAPP_geneticae}. 

%cambie un poco la estructura de resultados pero como para no reescribir la idea anterior le llame resultados 2
%% Los cap'itulos inician con \chapter{T'itulo}, estos aparecen numerados y
%% se incluyen en el 'indice general.
%%
%% Recuerda que aqu'i ya puedes escribir acentos como: 'a, 'e, 'i, etc.
%% La letra n con tilde es: 'n.


\chapter{Resultados}

En esta sección se muestran ejemplos de uso tanto del paquete geneticae como de Geneticae Shiny Web APP. 

\section{Paquete de R \emph{geneticae}}

Para instalar la versión del paquete publicada en CRAN: \textcolor{blue}{install.packages}(``geneticae"), mientras que la versión en desarrollo se debe instalar desde el repositorio de Github: devtools::\textcolor{blue}{install\_github}(``jangelini/geneticae"). 

Una vez instalado el paquete, se debe cargar en la sesion de R mediante el comando: \textcolor{blue}{library}(geneticae). 

Información detallada sobre las funciones del paquete geneticae se puede obtener mediante \textcolor{blue}{help}(package = ``geneticae"). La ayuda para una función, por ejemplo \textcolor{blue}{imputation}(), en una sesión R se puede obtener usando \emph{?imputation} o \textcolor{blue}{help}(imputation). La función \textcolor{blue}{browseVignettes}(``geneticae") permite obtener la viñeta del paquete, es decir una descripción el problema que está diseñado para resolver así como ejemplos de aplicación del mismo. 

Además, se encuentra disponible una página web (http://....) que contiene una breve descripción de la utilidad del paquete, las funciones que se incluyen en él, un tutorial de uso, un enlace de acceso a la shiny app, entre otra información.


\subsection{Conjuntos de datos en geneticae}
\label{subsec:datosejemplos}
El paquete geneticae proporciona dos conjuntos de datos que pueden utilizarse para ilustrar la metodología incluida para analizar los datos provenientes de EMA. 

\begin{itemize}[wide, nosep, labelindent = 0pt, topsep = 1ex, noitemsep,topsep=0pt]
\item \emph{yan.winterwheat dataset} \citep{Wright2020}: se cuenta con información sobre el rendimiento de 18 variedades de trigo de invierno cultivadas en nueve ambientes en Ontario en 1993. A pesar de que el experimento contaba con cuatro bloques o réplicas en cada ambiente, sólo el rendimiento medio para cada combinación de variedad y ambiente se encuentra disponible.\\

\begin{tcolorbox}[skin=bicolor,
    colframe=aurometalsaurus,colback=backcolour,colbacklower=white,
    width=1\linewidth,
    height=0.35\linewidth,
    boxsep=-3mm]
\begin{lstlisting}[linewidth=\columnwidth]
data(yan.winterwheat)
head(yanwinterwheat)
\end{lstlisting}

\tcblower\vskip-\baselineskip
\tcblower
\vspace{0.5cm}
\footnotesize\begin{verbatim}
##   gen  env yield
## 1 Ann BH93 4.460
## 2 Ari BH93 4.417
## 3 Aug BH93 4.669
## 4 Cas BH93 4.732
## 5 Del BH93 4.390
## 6 Dia BH93 5.178
\end{verbatim}
\end{tcolorbox}

\item \emph{plrv dataset} \citep{deMendiburu2020} se registró información sobre el rendimiento, el peso de planta y de la parcela de 28 genotipos en 6 localidades de Perú con el fin de estudiar la resistencia a PLRV (\emph{Patato Leaf Roll Virus}) causante del enrollamiento de la hoja. Cada clon fue evaluado tres veces en cada ambiente. \\

\begin{tcolorbox}[skin=bicolor,
    colframe=aurometalsaurus,colback=backcolour,colbacklower=white,
    width=1\linewidth,
    height=0.35\linewidth,
    boxsep=-3mm]
\begin{lstlisting}
data(plrv)
head(plrv)
\end{lstlisting}

\tcblower\vskip-\baselineskip
\tcblower
\vspace{0.5cm}
\footnotesize\begin{verbatim}
##   Genotype Locality Rep WeightPlant WeightPlot    Yield
## 1   102.18     Ayac   1   0.5100000       5.10 18.88889
## 2   104.22     Ayac   1   0.3450000       2.76 12.77778
## 3   121.31     Ayac   1   0.5425000       4.34 20.09259
## 4   141.28     Ayac   1   0.9888889       8.90 36.62551
## 5   157.26     Ayac   1   0.6250000       5.00 23.14815
## 6    163.9     Ayac   1   0.5120000       2.56 18.96296
\end{verbatim}
\end{tcolorbox} 
\end{itemize}
  
  
En las siguiente subsecciones se muestran las herramientas de análisis incluidas en el paquete utilizando el conjunto de datos \emph{yan.winterwheat}.  
  
\subsection{Modelo AMMI}

Para visualizar el efecto de IGA se utiliza el biplot GE obtenido del modelo AMMI. Este gráfico es posible obtenerlo utilizando la función \textcolor{blue}{rAMMI}(). Esta función requiere datos en formato largo, es decir, cada fila corresponde a una observación y cada columna a una variable (genotipo, ambiente, repetición (si existe) y fenotipo observado). Si cada genotipo ha sido evaluado más de una vez en cada ambiente, la media fenotípica para cada combinación de genotipo y ambiente se calcula internamente y luego se estima el modelo. Las variables adicionales que no se utilizarán en el análisis pueden estar presentes en el conjunto de datos. No se permiten valores perdidos pero se pueden imputar como se indica en la subsección \ref{subsec:metimp}. 

El biplot clásico para el conjunto de datos \emph{yan.winterwheat} se muestra en la figura \ref{fig:ammibip} junto con la sentencia utlizada para obtener el mismo. El primer argumento es el conjunto de datos de entrada, luego se indican los nombres de las columnas en las cuales se encuentra la información necesaria para aplicar la técnica y por último el biplot que se desea obtener que por defecto es el derivado del modelo AMMI clásico. Opcionalmente, el porcentaje de IGA explicado por el biplot se puede agregar como una nota al pie con el argumento \emph{footnote = T} así como un título con \emph{titles = T}. 

En este ejemplo, BH93, KE93 y OA93 son los ambientes que más contribuyen a la interacción ya que sus vectores son los de mayor magnitud. Los cultivares m12 y Kat presentan patrones de interacción similares (sus marcadores están próximos entre sí) y son muy diferentes de Ann y Aug, por ejemplo. La cercanía entre el cultivar Dia y el ambiente BH93 indica una fuerte asociación positiva entre ellos, lo que significa que BH93 es un ambiente extremadamente favorable para ese genotipo. Como los marcadores OA93 y Luc son opuestos, este ambiente es considerablemente desfavorable para ese genotipo. Por último, Cas y Reb están cerca del origen, lo que significa que se adaptan en igual medida a todos los ambientes.

\begin{tcolorbox}[skin=bicolor,
    colframe=aurometalsaurus,colback=backcolour,colbacklower=white,
    width=1\linewidth,
    height=0.82\linewidth,
    boxsep=-3mm]
\begin{lstlisting}
rAMMI(yan.winterwheat, genotype = ``gen", environment = ``env", 
      response = ``yield", type = ``AMMI", footnote = F, titles = F)
\end{lstlisting}
\tcblower\vskip-\baselineskip
\tcblower
\begin{figure}[H]
	\begin{center}
		\includegraphics[width=0.80\textwidth]{./Graficos/AMMI_biplot.png}
	\end{center}
	\caption{Biplot GE obtenido del modelo AMMI clásico basado en los datos de rendimiento de trigo de invierno obtenidos en Ontario en 1993. El 71,66\% de la variabilidad de la IGA se explica por los dos primeros términos multiplicativos. Los cultivares se muestran en letras minúsculas y los ambientes en mayúsculas. }
	\label{fig:ammibip}
\end{figure}
\end{tcolorbox}


Como se mencionó anteriormente, el modelo AMMI, en su forma estándar, asume que no hay valores atípicos presentes en los datos. Por lo tanto, en presencia de \emph{outliers} se debe utilizar alguna de las alternativas robustas propuestas por \citet{Rodriguesetal2016}, las cuales no se encuentran disponible en R hasta el momento. Sin embargo, dada la importancia práctica de este reciente avance metodológico, se incluyeron en la función \textcolor{blue}{rAMMI}(). Para obtener los biplots GE derivados de los modelos robustos se debe indicar en el argumento \emph{type} cuál de ellos se desea ajustar: ``rAMMI", ``hAMMI", ``gAMMI", ``lAMMI", ``ppAMMI".

Dado que el conjunto de datos de muestra \emph{yan.winterwheat} no presenta valores atípicos, las conclusiones obtenidas con biplots robustos no diferirán de las obtenidas con el biplot clásico \citep{Rodriguesetal2016}. Por lo tanto, no se presenta ninguna interpretación de los biplots robustos. \\


\subsection{Modelo de Regersión por Sitio}
\label{subsec:SREGpaquete}
Para visualizar conjuntamente el efecto de G e IGA \citet{Yanetal2000} propuso el biplot GGE mediante el cual se pueden abordar diversos aspectos relacionados con la evaluación de genotipos y ambientes. Para obtener dicho biplot en primer lugar se debe ajustar el modelo SREG mediante la función \textcolor{blue}{GGEmodel}(). Ésta es un \emph{wrapper} de \textcolor{blue}{GGEModel}() del paquete GGEBiplots \citep{Dumble2017}. Como en el caso de \textcolor{blue}{rAMMI}(), para poder utilizarla los datos deben presentarse en un formato largo y se permiten repeticiones o variables adicionales en el conjunto de datos. El rasgo fenotípico para cada combinación de genotipo y ambiente debe estar registrado, sino se debe recurrir previamente a alguna técnica de imputación para completar los datos (subsección  \ref{subsec:metimp}). 

La sentencia utilizada para ajustar el modelo GGE en el conjunto de datos \emph{yan.winterwheat} se muestra a continuación. El primer argumento de la misma consiste en el nombre del conjunto de datos y en los siguientes indican los nombres que reciben las columnas que contienen la información de los genotipos, ambientes y del rasgo fenotípico de interés. Por defecto, la función considera que no hay réplicas en el conjunto de datos, sin embargo, si existieran en el parámetro \emph{rep} se debe indicar el nombre de la columna con dicha información. Otros argumentos de dicha función son el método de centrado, de partición de los valores singulares (SVP de sus siglas en ingles \emph{Singular Value Partition}) y escalado. Por defecto los datos se centran utilizando la opción \emph{centering=``tester"} lo cual resulta en el modelo SREG, otro valor dará lugar a un modelo diferente. La elección del método de SVP no altera las relaciones o interacciones relativas entre los genotipos y los ambientes, aunque la apariencia del biplot será diferente (Yan 2002). El método de particion de los valores singulares centrado en los genotipos (\emph{SVP=``row"}) muestra la interrelación entre genotipos con mayor precisión, el enfocado a los ambientes (\emph{SVP=``column"}) es el más informativo de las interrelaciones entre los ambientes, mientras que el simétrico (\emph{SVP=``symmetrical"}) permite visualizar la magnitud relativa tanto de la variación de los genotipos como de los ambientes, por lo que se utiliza por defecto. Por último, se indica que los datos no se deben escalar con el parámetro \emph{scaling=``none"}. \\

\begin{tcolorbox}[colframe=aurometalsaurus,colback=backcolour,colbacklower=white,
   				width=1\linewidth,
    			height=0.12\linewidth,
    			boxsep=-3mm]
\begin{lstlisting}
GGE1 <- GGEmodel(yan.winterwheat, genotype = ``gen", environment = ``env", 
                 response = ``yield", rep = NULL,  centering = ``tester",
  				 scaling = ``none",  SVP = ``symmetrical")
\end{lstlisting}
\end{tcolorbox}


La salida de \textcolor{blue}{GGEModel}() es una lista con los siguientes objetos:
\begin{itemize}
\item coordgenotype: coordenadas para los genotipos en cada componente.
\item coordenviroment: coordenadas para los ambientes en cada componente.
\item eigenvalues: vector de autovalores para cada componente.
\item vartotal: varianza general.
\item varexpl: porcentaje de varianza explicado por cada componente.
\item labelgen: nombres de los genotipos.
\item labelenv: nombres de los ambientes.
\item axes: etiquetas de los ejes.
\item Data: datos escalados y centrados.
\item centering: método de centrado.
\item scaling: método de escala.
\item SVP: método de partición. 
\end{itemize}


Utilizando la salida de \textcolor{blue}{GGEmodel}(), la función 
\textcolor{blue}{GGEPlot}() crea numerosas vistas del biplot GGE que permiten dar respuesta a distintos objetivos de los fitomejoradores. En estos gráficos los cultivares se muestran en minúsculas y los ambientes en mayúsculas. El método de centrado, escalado y SVP se muestran en una nota al pie junto con el porcentaje de G + IGA explicado por los dos ejes al agregar el argumento \emph{footnote = T} y un título con \emph{titles = T}. \\

\textbf{Comparaciones simples utilizando GGE biplot}

El biplot básico se obtiene con el parámetro \emph{type = ``Biplot"} (Figura \ref{fig:ggebip}). En este ejemplo, el 78\% de la variabilidad de G e IGA se explica por los dos primeros términos multiplicativos. Los ángulos entre los marcadores de genotipos y entre los vectores ambientales son utilizados para interpretar el gráfico. Así, por ejemplo, Kat tiene un rendimiento por debajo de la media en todos los ambientes debido a su ángulo superior a $90^{\circ}$ con todos ellos. Por otro lado, Fun presenta un rendimiento superior a la media en todas las localidades excepto OA93 y KE93, como lo indican los ángulos agudos. La longitud de los vectores ambientales es una medida de la capacidad del ambiente para discriminar entre cultivos. 


\begin{tcolorbox}[skin=bicolor,
    colframe=aurometalsaurus,colback=backcolour,colbacklower=white,
    width=1\linewidth,
    height=0.82\linewidth,
    boxsep=-3mm]
\begin{lstlisting}
GGEPlot(GGE1, type = ``Biplot", footnote = F, titles = F)
\end{lstlisting}
\tcblower\vskip-\baselineskip
\tcblower
\begin{figure}[H]
	\begin{center}
		\includegraphics[width=0.80\textwidth]{./Graficos/GGE_biplot.png}
	\end{center}
	\caption{Biplot GGE basado en datos de rendimiento de trigo de invierno obtenido de Ontario en 1993. El método de partición de valores singulares utilizado es el simétrico (opción por defecto). El 78\% de la variabilidad de G + GE se explica por los dos primeros términos multiplicativos. Los cultivares se muestran en minúsculas y los entornos en mayúsculas. }
	\label{fig:ggebip}
\end{figure}
\end{tcolorbox} 

Los mejoradores quieren identificar los cultivares más adaptados a su área, es decir a un ambiente particular, por ejemplo OA93. Para esto, \citet{YanHunt2002} sugieren constituir un eje del ambiente de interés (OA93), trazando una recta que una el identificador del ambiente y el origen de coordenadas, y lo denominan eje OA93. Los genotipos se  clasifican en función del rendimiento en dicho ambiente de acuerdo con sus proyecciones, en la dirección indicada por el eje OA93 (Figura \ref{fig:selectEyG} A). Para obtener esta vista del biplot GGE, se indica la opción \emph{Selected Environment} en el argumento \emph{type} de la función y el ambiente a evaluar en el argumento  \emph{selectedE}. En este ejemplo, el cultivar de mayor rendimiento fue es Zav seguido por Aug, Ham hasta llegar al genotipo Luc, que es el de menor rendimiento en ese ambiente. El eje perpendicular al del ambiente de interés, separa los genotipos con rendimiento mayor al promedio, de Zav a Cas, de aquellos con valores inferior a la media, de Ema a Luc, en OA93.
 
En forma similar, el ambiente más adecuado para un cultivar es posible determinarlo graficando una línea que conecte el origen de coordenadas y el marcador del genotipo de interés, por ejemplo Kat, como se muestra en la figura \ref{fig:selectEyG} B \citep{YanHunt2002}. Los ambientes se clasifican a lo largo del eje del genotipo en la dirección indicada por la flecha. Para obtener este gráfico la opción  \emph{Selected Genotype} debe indicarse en el argumento \emph{type}, y el genotipo de interés en \emph{selectedG}. El eje perpendicular al del genotipo separa los ambientes en los que el cultivar presentó un rendimiento por debajo y por encima del promedio. En este ejemplo, Kat presentó un desempeño por debajo de la media en todos los ambientes estudiados. \\

\begin{tcolorbox}[skin=bicolor,
    colframe=aurometalsaurus,colback=backcolour,colbacklower=white,
    width=1\linewidth,
    height=0.92\linewidth,
    boxsep=-3mm]
\begin{lstlisting}
# Ranking de cultivares en el ambiente OA93
GGEPlot(GGE1, type = ``Selected Environment", selectedE = ``OA93", footnote = F, titles = F)

# Ranking de ambientes para cultivar Kat
GGEPlot(GGE1, type = ``Selected Genotype", selectedG = ``Kat", footnote = F, titles = F)
\end{lstlisting}
\tcblower\vskip-\baselineskip
\tcblower
\begin{figure}[H]
	\begin{center}
		\includegraphics[width=14cm]{./Graficos/SelectedGyE.png}
	\end{center}
	\caption{A: Ranking de cultivares en el ambiente OA93. B: Ranking de ambientes para cultivar Kat, basado en datos de rendimiento de trigo de invierno obtenido de Ontario en 1993. El método de partición de valores singulares utilizado es el simétrico (opción por defecto). El 78\% de la variabilidad de G + GE se explica por los dos primeros términos multiplicativos. Los cultivares se muestran en minúsculas y los entornos en mayúsculas.}
	\label{fig:selectEyG}
\end{figure}
\end{tcolorbox} 


También es posible comparar dos cultivares, por ejemplo Kat y Cas, vinculándolos con una línea y una perpendicular a la anterior (figura \ref{fig:comp2G}). Este biplot se obtiene con \emph{Comparison of Genotype} en el argumento \emph{type} y los genotipos a comparar en \emph{selectedG1} y \emph{selectedG2}. Cas fue más rendidor que Kat en todos los ambientes, ya que todos se ubican en el mismo lado de la línea perpendicular que Cas. 

\begin{tcolorbox}[skin=bicolor,
    colframe=aurometalsaurus,colback=backcolour,colbacklower=white,
    width=1\linewidth,
    height=0.82\linewidth,
    boxsep=-3mm]
\begin{lstlisting}
GGEPlot(GGE1, type = ``Comparison of Genotype", selectedG1 = ``Kat", selectedG2 = ``Cas", footnote = F, titles = F)
\end{lstlisting}
\tcblower\vskip-\baselineskip
\tcblower
\begin{figure}[H]
	\begin{center}
		\includegraphics[width=0.80\textwidth]{./Graficos/GGE_comparisonG1yG2.png}
	\end{center}
	\caption{comparación de los cultivares Kat y Cas. El método de partición de valores singulares utilizado es el simétrico (opción por defecto). El 78\% de la variabilidad de G + GE se explica por los dos primeros términos multiplicativos. Los cultivares se muestran en minúsculas y los entornos en mayúsculas. }
	\label{fig:comp2G}
\end{figure}
\end{tcolorbox}


\textbf{Identificación de mega-ambientes con GGE biplot}

La vista poligonal del biplot GGE, obtenida al indicar \emph{Which Won Where/What} en el argumento \emph{type}, proporciona un medio eficaz de visualización del patrón ``quíen ganó dónde"  de un conjunto de datos EMA (Figura \ref{fig:poligono}).  El polígono se obtiene uniendo los cultivares (fun, zav, ena, kat y luc) que se encuentran más alejados del origen de coordenadas, de modo que todos los restantes se encuentren contenidos en el polígono. La distancia de los cultivares respecto del origen de coordenadas, en sus respectivas direcciones, es una medida de la capacidad de respuesta a los ambientes. Los ubicados en los vértices son los más alejados, por lo tanto son los cultivares que más responden, mientras que los que se encuentran en el origen de coordenadas no responden en absoluto a los ambientes estudiados.

Las perpendiculares a los lados del polígono dividen al biplot en mega-ambientes, siendo el cultivar de mayor rendimiento en todos los ambientes que se encuentran en él aquel que se encuentra en el vértice de dicho sector. Por un lado, se observa que OA93 y KE93 conforman un mega-ambiente y que Zav es el mejor cultivar. Otro está formado por el resto de los ambientes, al cual llamaremos ME1 en futuros análisis, siendo Fun el que se encuentra en el vértice. En el sector con ena, kat y luc en los vértices del polígono no se observó ningún ambiente, lo cual indica que estos cultivares fueron los menos rendidores en algunos o todos los ambientes considerados.\\

\begin{tcolorbox}[skin=bicolor,
    colframe=aurometalsaurus,colback=backcolour,colbacklower=white,
    width=1\linewidth,
    height=0.82\linewidth,
    boxsep=-3mm]
\begin{lstlisting}
GGEPlot(GGE1, type = ``Which Won Where/What", footnote = F, titles = F)
\end{lstlisting}
\tcblower\vskip-\baselineskip
\tcblower
\begin{figure}[H]
	\begin{center}
		\includegraphics[width=0.80\textwidth]{./Graficos/GGE_whichwonwhere.png}
	\end{center}
	\caption{Vista poligonal del biplot GGE, que muestra qué cultivares presentaron mayor rendimiento en cada ambiente/mega-ambiente. El método de partición de valores singulares utilizado es el simétrico (opción por defecto). El 78\% de la variabilidad de G + GE se explica por los dos primeros términos multiplicativos. Los cultivares se muestran en minúsculas y los entornos en mayúsculas.}
	\label{fig:poligono}
\end{figure}
\end{tcolorbox}

\textbf{Evaluación de los cultivos dentro de un mega-ambientes con GGE biplot}

Una vez identificado los mega-ambientes, el siguiente paso es seleccionar cultivares dentro de cada uno de ellos. De acuerdo con la figura  \ref{fig:poligono}, zav es el mejor cultivar para los ambientes en uno de los mega-ambiente y fun para el otro. Sin embargo, los fitomejoradores no seleccionarán un único cultivar en cada mega-ambiente, sino que es necesario evaluar todos los cultivares con el fin de conocer su desempeño (rendimiento y estabilidad).  

El biplot GGE, particularmente utilizando el factor de partición de la descomposición en valores singulares enfocando en los genotipos, es decir utilizando el argumento \emph{SVP=``row"} en la función \textcolor{blue}{GGEmodel}(), proporciona un medio superior para visualizar tanto el rendimiento medio como la estabilidad de los genotipos (Figura \ref{fig:evaluacionG}). Esto se debe a que la unidad de ambos ejes para los genotipos es la unidad original de los datos. Además, dado que el interés radica en los genotipos y no en los ambientes, se indica con el argumento \emph{sizeEnv = 0} de la función \textcolor{blue}{GGEPlot}() para que no se los muestre en el gráfico.


La visualización del rendimiento medio y la estabilidad de los genotipos se logra dibujando una coordenada ambiental promedio (AEC, por sus siglas en inglés \emph{Average environment coordination}). Por ejemplo, la Figura \ref{fig:evaluacionG}  muestra el AEC para el megaambiente ME1 compuesto por los entornos BH93, EA93, HW93, ID93, NN93, RN93, WP93. Mientras que la abscisa representa el efecto de G la ordenada el de la IGA, que es una medida de la variabilidad o inestabilidad, asociada con cada genotipo. Una mayor proyección sobre la ordenada AEC, independientemente de la dirección, significa mayor inestabilidad. Fun fue claramente el cultivar de mayor rendimiento, en promedio, en este megaambiente, seguido por Cas y Har, y Kat fue el más pobre. Mientras que Rub y Dia son más variables y menos estables que otros cultivares, por el contrario, Cas, Zav, Reb, Del, Ari y Kar, fueron más estables. 

La Figura \ref{fig:evaluacionG} compara los cultivares con el ``ideal” que es el más rendidor y con estabilidad absoluta. Este cultivar ideal se usa como referencia, ya que rara vez existe. La distancia entre los cultivares y el ideal se puede utilizar como medida de conveniencia. Los círculos concéntricos ayudan a visualizar estas distancias. En el ejemplo, para el ME1, Fun es el más cercano al cultivo ideal, y por tanto el más deseable, seguido de Cas y Har, y Kat fue el más lejano. \\

\textbf{FALTAEL SIGNO \$ EN LA PRIMER LINEA DEL CODIGO QUE SIGUE...PERO ME TIRA ERROR O PONERLO CON FILTER???}

\begin{tcolorbox}[skin=bicolor,
    colframe=aurometalsaurus,colback=backcolour,colbacklower=white,
    width=1\linewidth,
    height=1.1\linewidth,
    boxsep=-3mm]
\begin{lstlisting}

ME1 <- yan.winterwheat[yan.winterwheat env %in% c(``BH93", ``EA93",``HW93", ``ID93",``NN93", ``RN93", ``WP93"), ]
                                                   
# Modelo SREG enfocando SVD en los genotipos
GGE_Gpartition <- GGEmodel(ME1, genotype = ``gen", environment = ``env", response = ``yield", SVP = ``row")

# Visualizacion del rendimiento medio y la estabilidad
GGEPlot(GGE_Gpartition, type = ``Mean vs. Stability", footnote = F, titles = F, sizeEnv = 0)

# Ranking de los genotipos respecto a uno ideal
GGEPlot(GGE_Gpartition, type = ``Ranking Genotypes", footnote = F, titles = F, sizeEnv = 0)

\end{lstlisting}
\tcblower\vskip-\baselineskip
\tcblower
\begin{figure}[H]
	\begin{center}
		\includegraphics[width=14cm]{./Graficos/MeanvsStability2.png}
	\end{center}
	\caption{A: Evaluación de los cultivares con base en el rendimiento promedio y la estabilidad y B: Clasificación de genotipos con respecto al genotipo ideal, basado en el método de patición de la descomposición en valores singulares enfocado en los genotipos.}
	\label{fig:evaluacionG}
\end{figure}
\end{tcolorbox}


\textbf{Evaluación de los ambientes con GGE biplot}

A pesar de que el objetivo principal de los EMA es seleccionar cultivares también es posible evaluar los ambientes. Esto incluye varios aspectos: (i) evaluar si la región objetivo pertenece a uno o más megaambientes; (ii) identificar mejores entornos de prueba; (iii) detectar ambientes redundantes que no brindan información adicional sobre cultivares; y (iv) determinar los ambientes que se pueden utilizar para la selección indirecta. Para ello, se enfoca la partición de los valores singulares en los ambientes al ajustar el modelo SREG (\emph{SVP = ``column"} en la función \textcolor{blue}{GGEmodel}()). 

En la figura \ref{fig:evaluacionE} los ambientes están conectados con el origen de coordenadas a través de vectores, permitiendo comprender las interrelaciones entre ellos.  Esta visualización del biplot GGE se obtiene indicando \emph{Relationship Among Environments} (Figura \ref{fig:evaluacionE}) en el parámetro \emph{type}. El coeficiente de correlación entre dos ambientes es aproximadamente el coseno del ángulo entre sus vectores. 
En este ejemplo se considera la relación entre los ambientes de ME1. El ángulo entre los vectores para los entornos NN93 y WP93 es de aproximadamente $10^{\circ}$ entre sus vectores; por lo tanto, están estrechamente relacionados; mientras que RN93 y OA93 presentan una correlación negativa débil ya que el ángulo es levemente mayor a $90^{\circ}$. El coseno de los ángulos no se traduce precisamente en coeficientes de correlación, ya que el biplot no explica toda la variabilidad en el conjunto de datos. Sin embargo, son lo suficientemente informativos como para comprender la interrelación entre los entornos de prueba. 

Si algunos de los ambientes tienen ángulos pequeños y, por lo tanto, están altamente correlacionados, la información sobre los genotipos obtenidos de estos ambientes debe ser similar. Si esta similitud es repetible a través de los años, estos ambientes son redundantes y por lo tanto, uno solo debería ser suficiente. Obtener la misma o mejor información utilizando menos ambientes reducirá el costo y aumentará la eficiencia de producción.


La capacidad de discriminación así como la representatividad respecto del ambiente objetivo, son medidas fundamentales para un ambiente. Si no tiene capacidad de discriminación, no proporciona información sobre los cultivares y, por lo tanto, carece de utilidad. A su vez, si no es representativo no sólo que carece de utilidad sino que también puede proporcionar información sesgada sobre los cultivares evaluados. Para visualizar estas medidas, se define una coordenada ambiental promedio (AEC mencionado anteriormente) y el ambiente ideal como el centro de un conjunto de círculos concéntricos (Figura \ref{fig:evaluacionE}). Para obtener este biplot se debe indicar \emph{Ranking Environments} en el argumento \emph{type} de \textcolor{blue}{GGEPlot}() (Figura \ref{fig:evaluacionE}). El ángulo entre el vector de un ambiente y el eje proporciona una medida de la representatividad. Por lo tanto, EA93 e ID93 son los más representativos, mientras que RN93 y BH93 son los menos representativos del ambiente promedio, cuando se analiza ME1. Por otro lado, para ser discriminativo debe estar cercano al ambiente ideal. HW93 es el ambiente más cercano al ideal y, por lo tanto, es el más deseable del ME1, seguido por EA93 e ID93. Por el contrario, RN93 y BH93 fueron los ambientes de prueba menos deseables de ME1. 

\begin{tcolorbox}[skin=bicolor,
    colframe=aurometalsaurus,colback=backcolour,colbacklower=white,
    width=1\linewidth,
    height=0.885\linewidth,
    boxsep=-3mm]
\begin{lstlisting}
# Modelo SREG enfocando SVD en los ambientes
GGE_Epartition <- GGEmodel(ME1, genotype=``gen", environment=``env", response=``yield", SVP=``column")

# Relacion entre ambientes
GGEPlot(GGE_Epartition, type = ``Relationship Among Environments", footnote = F, titles = F)

# Clasificacion de ambientes con respecto al ambiente ideal
GGEPlot(GGE_Epartition, type = ``Ranking Environments", footnote = F, titles = F)
\end{lstlisting}
\tcblower\vskip-\baselineskip
\tcblower
\begin{figure}[H]
	\begin{center}
		\includegraphics[width=14cm]{./Graficos/RelationshipEnvironments.png}
	\end{center}
	\caption{A: Relación entre ambientes y B: Clasificación de ambientes con respecto al ambiente ideal, basado en el escalado centrado en los genotipos.}
	\label{fig:evaluacionE}
\end{figure}
\end{tcolorbox}


\subsection{Métodos de imputación}
\label{subsec:metimp}

Una limitación importante de los modelos presentados anteriormente es que requieren que el conjunto de datos este completo, es decir que todos los genotipos sean evaluados en todos los ambientes. Por lo tanto, en el paquete se incluyen una serie de metodologías de imputación desarrolladas específicamente para datos genotipo-ambiente recientemente publicadas, algunas de las cuales no se encuentran disponible en R, para superar el problema de las observaciones perdidas. Entre los métodos incluidos se encuentran: ``EM-AMMI", ``EM-SVD", ``Gabriel",``WGabiel"  y ``EM-PCA", los cuales se indican en la opción \emph{type} de la función \textcolor{blue}{imputation}(). El formato requerido para el conjunto de datos de entrada es análogo al indicado en las otras funciones incluidas en el paquete. 

Para presentar un ejemplo, se eliminan algunas observaciones del conjunto de datos \emph{yan.winterwheat} ya que contaba con todos los registros completos:

\begin{tcolorbox}[skin=bicolor,
    colframe=aurometalsaurus,colback=backcolour,colbacklower=white,
    width=1\linewidth,
    height=0.15\linewidth,
    boxsep=-3mm]
\begin{lstlisting}
# Generando datos faltantes
yan.winterwheat [1,3] <- NA
yan.winterwheat [3,3] <- NA
yan.winterwheat [2,3] <- NA
\end{lstlisting}
\end{tcolorbox}

La imputación de valores perdidos con el método ``EM-AMMI" se puede realizar de la siguiente manera:

\begin{tcolorbox}[skin=bicolor,
    colframe=aurometalsaurus,colback=backcolour,colbacklower=white,
    width=1\linewidth,
    height=0.1\linewidth,
    boxsep=-3mm]
\begin{lstlisting}
imputation(yanwinterwheat, PC.nb = 2, genotype = ``gen", environment = ``env", response = ``yield", type = ``EM-AMMI")
\end{lstlisting}
\end{tcolorbox}

El resultado es la matriz con datos imputados en aquellas celdas vacías. 


\section{Geneticae Shiny Web App}

El objetivo de Geneticae Shiny Web APP es proporcionar una interfaz gráfica de usuario para el paquete geneticae de R descripto anteriormente, de modo que pueda ser utilizado por fitomejoradores y analistas sin experiencia previa en programación R. 

Es un software interactivo, no comercial y de código abierto, que ofrece una alternativa gratuita al software comercial disponible para analizar datos provenientes de ensayos multiambientales. Se encuentra disponible en un servidor gratiuto \url{https://geneticae.shinyapps.io/geneticae-shiny-web-app/} el cual tiene límite en el tiempo de uso, sin embargo, la APP será ubicada en el servidor de CONICET para su libre navegación. Además, se puede acceder a la misma desde la página web del instituto CEFOBI de CONICET \url{https://www.cefobi-conicet.gov.ar/bases-de-datos-y-programas/}.

En las subsecciones siguientes se presentará un ejemplo de cómo cargar y analizar datos con la aplicación.


\subsection{Preparación de un archivo de datos}

La APP requiere que los datos de entrada se encuentren en formato .csv, delimitados por comas, punto y coma o tabulaciones. Los nombres de las columnas pueden ubicarse en la primera fila del archivo (\emph{heading}). Los datos deben estar en formato largo, es decir, cada fila corresponde a una observación y cada columna a una variable (genotipo, ambiente, repetición (si existe) y fenotipo observado). Si cada genotipo ha sido evaluado más de una vez en cada ambiente, la media fenotípica requerida por el modelo SREG y AMMI para cada combinación de genotipo y ambiente se calcula internamente antes de ajustar dichos modelos. Las variables adicionales que no se utilizarán en el análisis pueden estar presentes en el conjunto de datos. No se permiten valores perdidos.

Los dos conjuntos de datos \emph{plrv} y \emph{yanwinterwheat} descriptos en la subsección \ref{subsec:datosejemplos} están disponibles en la pestaña \emph{Data $->$ Example datasets} y se pueden descargar en formato .csv (Figura \ref{fig:dataexample}) para poder seguir el tutorial de uso de la APP. El conjunto de datos \emph{yanwinterwheat} no tiene repeticiones, mientras que \emph{plrv} sí. 

 \begin{figure}[H]
	\begin{center}
		\includegraphics[width=0.8\textwidth]{./Graficos/www/exampledata.jpg}
	\end{center}
	\caption{(A) Plrv dataset (B) yanwinterwheat dataset }
	\label{fig:dataexample}
\end{figure}

En los ejemplos que continuan se utilizará el conjunto de datos \emph{yanwinterwheat}.

\textbf{Cargando un conjunto de datos en la APP}

El conjunto de datos que se analizará debe cargarse en la pestaña \emph{Data $->$ Upload data}. Por ejemplo, para importar el conjunto de datos \emph{yanwinterwheat}, se debe cargar el archivo .csv. Una vez cargado, se debe indicar que está delimitado por comas, que la primera fila contiene los nombres de cada variable (\emph{heading}) y los nombres de las columnas que continen la información del genotipo, ambiente y rasgo fenotípico (gen, env y rendimiento en este ejemplo). Si hay repeticiones disponibles, se debe especificar el nombre de la columna con dicha información; de lo contrario, el campo queda vacío. 

 \begin{figure}[H]
	\begin{center}
		\includegraphics[width=16cm]{./Graficos/www/Data.png}
	\end{center}
	\caption{Cargando el conjunto de datos \emph{yanwinterwheat} en Geneticae APP}
	\label{fig:fig431}
\end{figure}


\subsection{Análisis descriptivo}

Cualquier estudio debe comenzar con un análisis descriptivo del conjunto de datos. La pestaña \emph{Descriptive Analysis} proporciona algunas herramientas para esto, como  \emph{boxplot}, diagrama y matriz de correlación así como también gráficos de interacción.

Un \emph{boxplot} que compara el rasgo cuantitativo entre ambientes o genotipos puede ser de interés (Figura \ref{fig:figdesc1}). Las medidas de resumen utilizadas para su construcción se muestran de forma interactiva moviendo el mouse dentro del panel de la figura. Además, se puede descargar como un archivo .png o en forma interactiva (.html), haciendo clic en la cámara que aparece en la figura o en el botón Descargar, respectivamente. El usuario puede personalizar algunos aspectos del gráfico, como el color de las cajas y los nombres de los ejes. 

\begin{figure}[H]
	\begin{center}
		\includegraphics[width=16cm]{./Graficos/Boxplot.jpg}
	\end{center}
	\caption{Diagrama de caja de (A) genotipos y (B) ambientes para el conjunto de datos \emph{yanwinterwheat}.}
	\label{fig:figdesc1}
\end{figure}

Los coeficientes de correlación de Pearson o Spearman entre genotipos se pueden mostrar como un gráfico o una matriz (Figura \ref{fig:figdesc2}). Gráficamente, las correlaciones positivas se muestran en azul y las negativas en rojo, mientras que la intensidad del color y el tamaño del círculo son proporcionales a la magnitud de los coeficientes de correlación. La gráfica de correlación se puede descargar en formato .png. Se observan altas correlaciones entre el rendimiento de los genotipos estudiados. 

\begin{figure}[H]
	\begin{center}
		\includegraphics[width=16cm]{./Graficos/correlacion.jpg}
	\end{center}
	\caption{Gráfico de correlación (A) y matriz (B) entre genotipos yanwinterwheat dataset }
	\label{fig:figdesc2}
\end{figure}


Dado que IGA genera respuestas genotípicas diferenciales en diferentes ambientes, lo que complica selección de los mejores cultivares, una gráfico de interacción puede ser de interés (Figura \ref{fig:figdesc3}). El cambio en el efecto genotípico a través de los ambientes se muestra en la figura \ref{fig:figdesc3} A, mientras que el cambio en el efecto ambiental a través de los genotipos en la figura \ref{fig:figdesc3} B. Del mismo modo que el \emph{boxplot} es un gráfico interactivo, por lo que es posible descargarla en formatos .HTML o .png con el botón Descargar o haciendo clic en la cámara, respectivamente. Además, el usuario puede personalizar los nombres de los ejes. En este ejemplo se pueden ver inconsistencias en el desempeño de genotipos en diferentes ambientes. 


\begin{figure}[H]
	\begin{center}
		\includegraphics[width=16cm]{./Graficos/interaction.jpg}
	\end{center}
	\caption{Gráfico de interacción para (A) ambientes a través de genotipos y (B) genotipos a través de entornos del conjunto de datos de \emph{yanwinterwheat}.}
	\label{fig:figdesc3}
\end{figure}


\subsection{Modelo de regresión por sitio}

\emph{Geneticae Shiny Web App} permite generar las vistas del biplot GGE presentados en la subsección \ref{subsec:SREGpaquete} mediante la pestaña \emph{GGE Biplot}. Del mismo modo que en el paquete geneticae, los cultivares se presentan en minúsculas y los ambientes en mayúsculas. Dado que el modelo SREG requiere una única observación para cada combinación de genotipo y ambiente, si hay repeticiones, el valor fenotípico promedio se calcula automáticamente antes de ajustar el modelo. No se permiten valores perdidos. 

Se debe seleccionar el método de partición de los valores singulares (\emph{SVP type}) sin embargo, como se mencionó anteriormente esta elección no altera las relaciones o interacciones relativas entre genotipos y ambientes, aunque la apariencia del biplot será diferente (Yan, 2002). La opción simétrica permite la comparación tanto de genotipos como de ambientes (opción por \emph{default}); \emph{Genotype-Focused} muestra la interrelación entre genotipos con mayor precisión que cualquier otro método, y \emph{Environment-Focused} es la que más informa sobre las interrelaciones entre ambientes. Una nota a pie del gráfico que indica que el método de centrado, que será siempre \emph{tester-center} para obtener el biplot GGE, que no se aplica ninguna escala a los datos, el método SVP seleccionado por el usuario y el porcentaje de variación de G e IGA explicado por los dos ejes puede ser agregado. A su vez, el título del gráfico, los ejes y los nombres de los mismos se pueden configurar para que aparezcan o no. Por último, ciertos atributos estilísticos de dichos gráficos se pueden personalizar como el color y tamaño de los marcadores de genotipos y ambientes, y además los gráficos pueden ser descargados.  

El biplot básico, la vista del biplot GGE que muestra los cultivares más adecuados para un ambiente particular (OA93), los ambientes más adecuados para un genotipo (Kat), la comparación de dos genotipos (Cas y Kat) y la vista del poligono se pueden obtener como se indica en la figura \ref{fig:ggebip1}, donde el escalado es el simétrico (\emph{SVP type $->$ symmetrical}) y las opciones de \emph{plot type} son \emph{Biplot,
Selected Environment, Selected Genotype, Comparison of Genotype} y \emph{Which Won Where/What}, respectivamente. Al indicar \emph{Selected Environment} el ambiente de interés se debe especificar, de igual modo cuando se utiliza \emph{Selected Genotype} y 
\emph{Comparison of Genotype} se debe señalar cuál es el genotipo a analizar.

\begin{figure}[H]
	\begin{center}
		\includegraphics[width=16cm]{./Graficos/www/GGE_biplotAPP1.png}
	\end{center}
	\caption{Boxplot de genotipos a través de los ambientes para el conjunto de datos Plrv}
	\label{fig:ggebip1}
\end{figure}

La selección de cultivares dentro de cada megaambiente se realiza con la partición de valores singulares enfocada en los genotypos (\emph{SVP type $->$ genotype-focused}), y los tipos de gráficos que se pueden realizar son: \emph{Mean vs. Stability} que permite la visualización de la media y estabilidad de genotipos y \emph{Ranking Genotypes} que compara las cultivares con el ``ideal" (Figura \ref{fig:ggebip2}). Dado que estos análisis son propios de cada megaambiente, al indicar alguno de estas vistas del biplot GGE se tendrá que señalar cuales son los ambientes que forman el megaambiente de interés. 


\begin{figure}[H]
	\begin{center}
		\includegraphics[width=16cm]{./Graficos/www/GGE_biplotAPP2.png}
	\end{center}
	\caption{Boxplot de genotipos a través de los ambientes para el conjunto de datos Plrv}
	\label{fig:ggebip2}
\end{figure}

Por último, para el análisis de los ambientes de cada megaambiente se utiliza el método de partición de valores singulares centrado en los ambientes (\emph{SVP type $->$ environment-focused}). Para comprender las interrelaciones entre ellos el tipo de gráfico \emph{Relationship Among Environments} se debe seleccionar y para visualizar la dapacidad de discriminación y representatividad 
\emph{Ranking Environments} (Figura \ref{fig:ggebip3}). Dado que estos análisis son propios de cada megaambiente, al indicar alguno de estas vistas del biplot GGE se tendrá que señalar cuales son los ambientes que forman el megaambiente de interés. 


\begin{figure}[H]
	\begin{center}
		\includegraphics[width=16cm]{./Graficos/www/GGE_biplotAPP3.png}
	\end{center}
	\caption{Boxplot de genotipos a través de los ambientes para el conjunto de datos Plrv}
	\label{fig:ggebip3}
\end{figure}


\subsection{modelo AMMI}

La pestaña \emph{AMMI Biplot} crea el biplot GE, en el que los cultivares se muestran en minúsculas y los entornos en mayúsculas. Dado que las alternativas clásica y robustas requieren una única observación para cada combinación de genotipo y ambiente, si hay repeticiones, el valor promedio fenotípico se calcula automáticamente antes de ajustar el modelo. No se permiten valores perdidos. Al igual que en el biplot de GGE, una nota a pie de página que indica que el porcentaje de variación de IGA explicado por los dos ejes, el gráfico de título, los ejes y los nombres de los mismos se pueden configurar para que aparezcan o no. Además, el color y tamaño del marcador de genotipos y ambientes pueden ser personalizados por el usuario. Los biplots pueden ser descargados.

Por ejemplo, para obtener el biplot GE derivado del modelo AMMI clásico se debe indicar
AMMI en \emph{plot type} (Figura ...). En caso de contar con \emph{outliers} alguna de las alternativas robustas (rAMMI, hAMMI, gAMMI, lAMMI o ppAMMI) se debe especificar.
 

\begin{figure}[H]
	\begin{center}
		\includegraphics[width=16cm]{./Graficos/AMMI_GE.png}
	\end{center}
	\caption{AMMI}
	\label{fig:fig4313}
\end{figure}


\subsection{Ayuda}

En la pestaña \emph{Help} se presenta información general, un tutorial y un video sobre  cómo utilizar la APP.

%\include{capitulo3}
\chapter{Conclusión}

En etapas avanzadas de los programas de mejoramiento vegetal, comúnmente se llevan a cabo EMA que consisten en la evaluación de un conjunto de variedades en múltiples ambientes. Un análisis adecuado de la información de los EMA es indispensable para el éxito del programa de mejoramiento genético de los cultivos. Dado que, la metodología utilizada se encuentra en constante desarrollo, muchas de ellas no se encuentran disponibles en programas comerciales. 

En este trabajo se muestra un flujo de trabajo reproducible para la construcción de paquetes de R que puede utilizarse de ejemplo para el desarrollo de nuevos paquetes. En particular se creó el paquete de R \emph{geneticae} que es de gran utildad para el análisis de datos provenientes de EMA por incluir metodología recientemente publicada además de la reunir las funciones más útiles para tal fin. En el momento de la escritura de este informe, pasaron 3 semanas desde su publicación en el repositorio CRAN y cuenta con más de 400 descargas, a pesar de que aún no se ha hecho difusión del paquete. 

Por otro lado, dado que el uso del software puede resultar dificultoso para aquellos analistas no familiarizados con la programación, se crea una aplicación web Shiny de libre acceso mediante conexión a internet que permite realizar los principales análisis implementados en el paquete sin necesidad de escribir líneas de código.

Se plantea para un futuro, continuar con la inclusión de metodología frecuentemente utilizada así como aquellas que se vayan desarrollando en el contexto de datos provenientes de EMA tanto en el paquete como en la aplicación web Shiny. 





%\chapter*{Perspectivas futuro}

permitir imputar en la aplicación, permitir hacer graficas por mega ambiente





%%\appendix
%% Cap'itulos incluidos despues del comando \appendix aparecen como ap'endices
%% de la tesis.
%%%% Los cap'itulos inician con \chapter{T'itulo}, estos aparecen numerados y
%% se incluyen en el 'indice general.
%%
%% Recuerda que aqu'i ya puedes escribir acentos como: 'a, 'e, 'i, etc.
%% La letra n con tilde es: 'n.


%\setcounter{section}{1}
\chapter{Hoja de referencia Shiny}
\includepdf[pages={1-2},scale=.78,pagecommand={},angle=90]{shiny-spanish.pdf}

%%%% Los cap'itulos inician con \chapter{T'itulo}, estos aparecen numerados y
%% se incluyen en el 'indice general.
%%
%% Recuerda que aqu'i ya puedes escribir acentos como: 'a, 'e, 'i, etc.
%% La letra n con tilde es: 'n.


%\setcounter{section}{1}
\chapter{Guías para usuario de Geneticae APP}


%%\include{apendiceC}

%% Incluir la bibliograf'ia. Mirar el archivo "biblio.bib" para m'as detales
%% y un ejemplo.

\bibliography{biblio}

\end{document}
