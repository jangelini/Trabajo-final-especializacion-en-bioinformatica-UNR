\chapter{Introducción}

A lo largo de la historia de la agricultura, el hombre ha desarrollado el mejoramiento vegetal en forma sistemática y lo ha convertido en un instrumento esencial para incrementar la producción agrícola en términos de cantidad, calidad y diversidad. Las variedades mejoradas son el resultado del trabajo llevado a cabo en los programas de fitomejoramiento, los cuales se extienden a lo largo de varios años y requieren cuantiosas inversiones. En etapas avanzadas de estos programas, comúnmente se llevan a cabo ensayos multiambientales (EMA) de comparación de rendimientos, donde un conjunto de variedades se evalúan en múltiples ambientes. Éstos son esenciales ya que además de los efectos genotípicos y ambientales, se puede detectar un efecto adicional, el proporcionado por la interacción entre ambos \citep{CruzRegazzi1997}. La interacción genotipo ambiente (IGA) es la respuesta diferencial de los genotipos a través de un rango de ambientes y es considerada, casi unánimemente por los fitomejoradores, como el principal factor limitante para la selección de cultivares superiores, disminuyendo la eficiencia de los programas de mejoramiento \citep{Crossaetal1990, CruzMedina1992, KangMagari1996}. Cuando los ambientes son muy diferentes, la IGA usualmente gana importancia porque cambia el rango de las líneas de mejoramiento. \citet{GauchZobel1997} explicaron que si no hubiera interacción, una sola variedad o híbrido rendirían al máximo en todo el mundo, además los materiales podrían evaluarse en un solo lugar y proporcionarían resultados universales.

\citet{Peto1982} ha distinguido las interacciones cuantitativas, conocidas también como sin cambio de rango o \emph{no crossover interaction} (NCOI), de las cualitativas, denominadas a su vez como con cambio de rango o \emph{crossover interaction} (COI). Cuando dos genotipos $G_1$ y $G_2$ tienen una respuesta diferencial en dos ambientes, se dice que la IGA es del tipo COI si hay cambios en el orden de los genotipos según su rendimiento (Figura \ref{fig:fig11}(A)) y del tipo NCOI si su ordenamiento permanece sin cambios (Figura  \ref{fig:fig11}(B)). Por otro lado, se dice que la IGA es inexistente cuando los genotipos responden de manera similar en ambos ambientes (Figura \ref{fig:fig11}(C)). 


\begin{figure}[h]
\begin{center}
\includegraphics[width=14cm]{./Graficos/interac}
\end{center}
\caption{  Representación gráfica de tipos de interacción genotipo - ambiente: (A) \emph{crossover}, (B) \emph{no crossover} y (C) sin interacción.}
\label{fig:fig11}
\end{figure}


Los estudios de adaptabilidad y estabilidad fenotípica tienen como objetivo analizar el comportamiento de los materiales en diferentes ambientes, interesándose en la respuesta diferencial de los genotipos a la variación ambiental \citep{Borem2001}. En general, se ha aceptado que, a mayor variabilidad genética de una especie/población, mayor será su estabilidad sobre el ambiente. \citet{AllardBradshaw1964} indican que una población podría estar conformada por individuos diferentes, cada uno de ellos adaptados a un rango de ambientes, o conformada por individuos semejantes, pero cada uno adaptado a un rango de ambientes. La adaptabilidad se refiere a la habilidad del genotipo de tener buen desempeño (por ejemplo; rendimientos altos) con respecto a determinadas condiciones ambientales.  Conceptualmente la adaptabilidad es dividida en amplia y específica. La primera refiere a cultivares adaptados a una amplia red ambiental, y la segunda a una adaptación regional o mega-ambiente. Por otro lado, también la estabilidad es conceptualizada. Según \citet{Becker1981} existen dos tipos de estabilidad. Una denominada estática con sentido homeostático, y otra llamada dinámica o agronómica. La primera, caracteriza al desempeño de materiales que muestran variancia cero entre ambientes y no es deseable por los mejoradores ya que los materiales no responden a la mejora ambiental. La segunda, considera un material estable si su performance es buena respecto al potencial del ambiente. La estabilidad estática puede asociarse a una escasa IGA, en términos de ecovalencia, o falta de interacción, y la estabilidad dinámica con la interacción simple o NCOI. La COI es consecuencia del comportamiento impredecible de los materiales y es, por lo general, una limitante seria en la selección de genotipos.

La IGA es útil si el objetivo del programa es el desarrollo de materiales con adaptación especifica. Es decir, la identificación de genotipos que tienen un comportamiento destacado en una determinada región o mega-ambiente, lo que permite señalar nichos ambientales propicios para una mayor productividad y calidad. Por el contrario, la COI complica el proceso de selección ya que disminuye la correlación de los valores fenotípicos entre los ambientes, dificultando la identificación de aquellos genotipos con adaptación amplia.

Un análisis adecuado de la información de los EMA es indispensable para el éxito del programa de mejoramiento genético de los cultivos. El rendimiento medio de los genotipos en los ambientes es un indicador suficiente del desempeño de ellos sólo en ausencia de IGA \citep{YanKang2003}. Sin embargo, la aparición de IGA es inevitable y no basta con la comparación de las medias de los genotipos, sino que se debe recurrir a una metodología estadística más aporopiada. Las más difundidas para analizar los datos provenientes de EMA se basan en modificaciones de los modelos de regresión, análisis de variancia (ANOVA, siglas en inglés de \emph{analysis of variance}) y técnicas de análisis multivariado. 

Particularmente, para el estudio de la IGA y los análisis que de ella se derivan, dos modelos multiplicativos han aumentado su popularidad entre los fitomejoradores como herramientas de análisis gráfico: el modelo de los efectos principales aditivos e interacción multiplicativa (AMMI, siglas en inglés de \emph{Additive Main effects and Multiplicative Interaction}) \citep{Gauch1988, Gauch1992} y el de regresión por sitio (SREG, siglas en inglés de \emph{Site Regression model}) \citep{Corneliusetal1996, CrossaCornelius1997}. Estos modelos se ajustan en dos etapas. Primero, se realiza un ANOVA para obtener estimaciones de los efectos principales aditivos de ambientes y genotipos (G) en AMMI y sólo de los ambientes en SREG. En segundo lugar, los residuos del ANOVA se ordenan en una matriz con genotipos en las filas y ambientes en las columnas y se aplica una descomposición en valores singulares (DVS), representando los patrones de IGA presentes en los residuos en AMMI y de G e IGA conjuntamente en SREG. El resultado de los dos primeros términos multiplicativos de la DVS a menudo se presentan en un biplot llamado GE (genotipo-ambiente, siglas en inglés de \emph{Genotype-Environment}) para el modelo AMMI \citep{Zobel1988} y GGE (genotipo más genotipo-ambiente, siglas en inglés de \emph{Genotype plus Genotype-Environment}) para SREG \citep{Yanetal2000}. Sin embargo, estos modelos no siempre son lo suficientemente eficientes para analizar la estructura de datos provenientes de EMA de programas de mejoramiento vegetal \citep{deOliveira2016, Jarquin2016, Hadaschetal2018}. Por un lado, tienen serias limitaciones frente a información faltante y, a pesar de que los EMA están diseñados para que todos los genotipos se evalúen en todos los ambientes, la presencia de valores perdidos es muy común \citep{Woyann2017, Aguate2019}. Esto ocurre, por ejemplo, debido a errores de medición o destrucción de plantas por presencia de animales, inundaciones o problemas durante la cosecha, además de la dinámica propia de las evaluaciones en las que se incorporan y se descartan genotipos debido a su mal desempeño \citep{HillRosenberg1985}. Numerosas metodologías de imputación se han estado desarrollando en los ultimos años para solventar esta limitación \citep{Alarconetal2010, Alarconetal2014, JosseHusson2016, Alarconetal2020}. Por otro lado, ambos modelos son sensibles a la presencia de observaciones atípicas, lo cual es una regla más que una excepción cuando se consideran datos reales. Para superar esta fragilidad, recientemente distintas metodologías robustas se han desarrollado para el modelo AMMI \citep{Rodriguesetal2016}. 

En este contexto, el análisis de datos provenientes de EMA requiere metodología estadística cuyas rutinas informáticas no se encuentran disponibles en programas comerciales debido a su reciente desarrollo o bien se deben utilizar varios de ellos para cumplir un único objetivo. Esto último genera el inconveniente de tener que disponer de todos los programas necesarios para los distintos análisis, atender los requerimientos de formatos de datos usados por cada uno de ellos y comprender los diversos tipos de salidas en las que se presentan los resultados obtenidos. Además, los costos de las licencias algunos programas pueden resultar muy elevados. 

Ante estas dificultades, una alternativa para el análisis es el empleo de algún lenguaje de programación de distribución libre y gratuita, que le confiera al analista la flexibilidad necesaria para cumplir con su objetivo. En este contexto, R es uno de los lenguajes de programación desarrollados para el análisis de datos de mayor uso en la actualidad. R es un software de uso libre y distribuido bajo los términos de la \emph{General Public Licence}. Este programa se descarga de un repositorio mantenido por \emph{The R Foundation for Statistical Computing} conocido como CRAN (\emph{Comprehensive R Archive Network}), en el cual también se encuentran disponibles miles de paquetes adicionales que consisten en conjuntos de funciones desarrolladas con fines específicos que se distribuyen con un protocolo determinado, garantizando su correcto funcionamiento. Cualquier desarrollador puede producir su propio paquete y publicarlo en CRAN, siempre que cumpla con los requisitos establecidos y pase correctamente por los procedimientos de control. Además, hay paquetes que pueden obtenerse de otros repositorios como GitHub, Bioconductor, rOpenSci, entre otros. 

R es propicio para el análisis de datos de EMA puesto que se ha desarrollado metodología específica para este entorno computacional. Algunos de los paquetes desarrollados para tal fin son: \emph{agricolae} \citep{deMendiburu2020}, \emph{gge} \citep{WrightLaffont2021}, \emph{GGEBiplots} \citep{Dumble2017} y \emph{metan} \citep{Olivoto2020}. \citet{FrutosGalindoLeiva2013} desarrollaron el paquete \emph{GGEBiplotGUI} que consistía en una implementación computacional interactiva para el análisis de datos EMA que requería un mínimo conocimiento del lenguaje R (instalar, cargar librerías e importar bases de datos). Sin embargo, en 2021 fue archivado de CRAN debido a falta de mantenimiento (sigue disponible en GitHub).


A pesar de las ventajas del uso de R, el análisis de datos de EMA en dicho software presenta algunos desafíos. Por un lado, existen numerosos paquetes con funcionalidad afín que hay que identificar cómo combinar adecuadamente. Por otro lado, el software puede resultar dificultoso para aquellos analistas no familiarizados con la programación. Atendiendo a estas dos necesidades, se crea un paquete que incluya metodología recientemente publicada y reúna las funciones más útiles a fin de solventar la primera de ellas. Para la segunda, se crea una aplicación web Shiny de libre acceso mediante conexión a internet que permita realizar los principales análisis implementados en el paquete sin necesidad de escribir líneas de código. 
