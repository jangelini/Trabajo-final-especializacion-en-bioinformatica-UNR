
\chapter*{Resumen}

Las variedades mejoradas de cultivos vegetales son el resultado del trabajo de desarrollo genético llevado a cabo en los programas de fitomejoramiento, los cuales se extienden a lo largo de varios años y requieren cuantiosas inversiones. En etapas avanzadas, los ensayos multiambientales (EMA), que comprenden experimentos en múltiples ambientes, son herramientas fundamentales para incrementar la productividad y rentabilidad de los cultivos. La vigencia comercial de las variedades puede extenderse durante varias décadas, por lo que su elección es crítica para que el productor evite pérdidas económicas por malas campañas y el suministro al mercado sea constante. Consecuentemente, un análisis adecuado de la información de los EMA es indispensable para asegurar el éxito del programa de mejoramiento de cultivos. Actualmente, R es uno de los lenguajes de programación más utilizados para el análisis de datos debido a su distribución como software libre y a la gran variedad de herramientas que ofrece. Sin embargo, los mejoradores que no están familiarizados con la programación tienden a utilizar programas que responden a instrucciones por menú en lugar de escribir líneas de código, a pesar de los costos económicos derivados del pago de sus licencias. Aquellos que sí tienen afinidad con el uso de código para el análisis de datos se enfrentan con dificultades a la hora de identificar las herramientas apropiadas entre el gran número de instrumentos disponibles. Por lo tanto, en este trabajo se presenta el desarrollo de dos herramientas informáticas para asistir en el análisis de datos provenientes de EMA. Por un lado, se creó un nuevo paquete de R que incluye metodología recientemente publicada que no se encuentra disponible en el software y al mismo tiempo reúne todas aquellas de mayor utilidad, de modo que aquellos usuarios que posean un manejo del lenguaje puedan simplificar su tarea. Por otro lado, se confeccionó una interfaz gráfica de usuario mediante una aplicación web Shiny que permite realizar los principales análisis implementados en el paquete sin necesidad de programar y se encuentra publicada en internet para su libre acceso.

\textbf{Palabras Clave}: análisis estadístico, ensayos multiambientales, interfaz gráfica, lenguaje R, programación.

