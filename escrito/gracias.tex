%% Las secciones del "prefacio" inician con el comando \prefacesection{T'itulo}
%% Este tipo de secciones *no* van numeradas, pero s'i aparecen en el 'indice.
%%
%% Si quieres agregar una secci'on que no vaya n'umerada y que *tampoco*
%% aparesca en el 'indice, usa entonces el comando \chapter*{T'itulo}
%%
%% Recuerda que aqu'i ya puedes escribir acentos como: 'a, 'e, 'i, etc.
%% La letra n con tilde es: 'n.

\chapter*{Agradecimientos}
\begin{spacing}{1}

\emph{En este trabajo final, directa o indirectamente, participaron muchas personas a las que les quiero agradecer.}

\emph{En primer lugar al Dr. Gerardo Cervigni por haberme propuesto realizar la Especialización Bioinformática, compartir su conocimiento y experiencia a lo largo de todo el proceso, contagiando su pasión, entusiasmo y energía. }

\emph{Al Mgs. Marcos Prunello por acompañarme en el desarrollo del trabajo final, por su dedicación, sus consejos y su ejemplo que me incentiva a superarme como profesional. Sin su confianza, apoyo y atención, este trabajo no hubiera sido posible. No sólo me enriquecí en lo académico sino también con la amistad que pudimos forjar. }

\emph{Al Centro Computacional del Centro Científico Teconológico de Rosario, miembro del Sistema Nacional de Computación de Alto Rendimiento, por la predisposición, asesoramiento e instalación de los recursos adicionales necesarios para este trabajo. }

\emph{Al Dr. Sergio Arciniegas Alarcón por su predisposición en la inclusión de los avances metodológicos realizados por su equipo de investigación en este trabajo.}

\emph{A mis compañeros de la Especialización, por las largas horas de cursos, mates y almuerzos. En especial, a Jor y Lu, por el aliento en todo momento, por compartir excelentes momentos y porque gracias a la ayuda de ambas he podido entender cosas que no habría podido sola.}

\emph{A los docentes de la Especialización en Bioinformática por su dedicación y paciencia para enseñarle a alumnos provenientes de las más diversas áreas esta hermosa combinación entre Biología, Informática y Estadística.}

\emph{A mis padres por el amor y apoyo incondicional y por el esfuerzo  de  dedicar  sus  vidas  a  brindarnos  a mi hermano y a mí la  posibilidad  de construir nuestros futuros. A mi hermano, por su cariño, apoyo, acompañamiento y sentido del humor. A Otto, por su incomparable mezcla de amor y comprensión, por darme fuerzas en los momentos de debilidad y por alentarme a seguir a pesar de todo. A Segundo, Mia y Kalita, por su hermosa compañía día a día.}

\emph{Por último, pero no menos importante, a Gaby y Euge mis compañeras de CEFOBI, por acompañarme en las partes más empedradas del camino, por compartir las risas y las  lágrimas, por su amistad y consejos. No hubiese alcanzado mucho de mis logros sin su ayuda, compañía y aliento en todo momento.}
\end{spacing}



%% Por si alguien tiene curiosidad, este "simp'atico" agradecimiento est'a
%% tomado de la "Tesis de Lydia Chalmers" basada en el universo del programa
%% de televisi'on Buffy, la Cazadora de Vampiros.
%% http://www.buffy-cazavampiros.com/Spiketesis/tesis.inicio.htm
