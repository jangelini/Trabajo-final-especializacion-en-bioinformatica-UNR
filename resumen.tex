%% Las secciones del "prefacio" inician con el comando \prefacesection{T'itulo}
%% Este tipo de secciones *no* van numeradas, pero s'i aparecen en el 'indice.
%%
%% Si quieres agregar una secci'on que no vaya n'umerada y que *tampoco*
%% aparesca en el 'indice, usa entonces el comando \chapter*{T'itulo}
%%
%% Recuerda que aqu'i ya puedes escribir acentos como: 'a, 'e, 'i, etc.
%% La letra n con tilde es: 'n.

\chapter*{Resumen}

\thispagestyle{empty}

Los programas informáticos se han convertido, hoy en día, en una herramienta básica utilizada por el análisis estadístico como apoyo fundamental a la hora de realizar diferentes operaciones y para facilitar una mayor comodidad a los usuarios. Actualmente, R es uno de los programas más utilizados debido a su potencia y a su distribución como software libre. 



\textbf{Palabras Clave:}

%% Por si alguien tiene curiosidad, este "simp'atico" agradecimiento est'a
%% tomado de la "Tesis de Lydia Chalmers" basada en el universo del programa
%% de televisi'on Buffy, la Cazadora de Vampiros.
%% http://www.buffy-cazavampiros.com/Spiketesis/tesis.inicio.htm
