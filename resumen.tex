%% Las secciones del "prefacio" inician con el comando \prefacesection{T'itulo}
%% Este tipo de secciones *no* van numeradas, pero s'i aparecen en el 'indice.
%%
%% Si quieres agregar una secci'on que no vaya n'umerada y que *tampoco*
%% aparesca en el 'indice, usa entonces el comando \chapter*{T'itulo}
%%
%% Recuerda que aqu'i ya puedes escribir acentos como: 'a, 'e, 'i, etc.
%% La letra n con tilde es: 'n.

\chapter*{Resumen}

%\thispagestyle{empty}
Las variedades mejoradas son el resultado del trabajo de desarrollo genético llevado a cabo en los programas de fitomejoramiento, los cuales se extienden a lo largo de varios años y requieren cuantiosas inversiones. En etapas avanzadas, los ensayos multiambientales (EMA), que comprenden experimentos en múltiples ambientes, son herramientas fundamentales para incrementar la productividad y rentabilidad de los cultivos. La vigencia comercial de las variedades puede extenderse durante varias décadas, por lo que su elección es crítica para que el productor evite pérdidas económicas por malas campañas y el suministro al mercado sea constante. Consecuentemente, un análisis adecuado de la información de los EMA es indispensable para que el programa de mejoramiento de los cultivos sea efiecaz. Los programas informáticos se han convertido, hoy en día, en una herramienta esencial para el análisis de datos. Actualmente, R es uno de los lenguajes de programación más utilizados debido a su potencia y a su distribución como software libre. Sin embargo, existen numerosos paquetes de R lo cual provoca que no sea sencillo encontrar uno que sea útil para un determinado propósito sino que se debe recurrir a varios de ellos. Por otro lado, los mejoradores  utilizan programas que responden a instrucciones por menú en lugar de escribir lineas de código a pesar de los costos de los mismos y de ser mas restrictivos en cuanto a los análisis que permiten llevar a cabo. En este sentido el paquete Shiny que permite crear una interfaz gráfica entre R y el usuario, pudiendo acercar la potencia de R a todo tipo de usuarios. Por lo tanto, en el presente trabajo se desarrolló en primer lugar un paquete de R que le permita analizar los datos provenientes de EMA a aquellos usuarios que posean un manejo del lenguaje de programación y, en segundo lugar, aplicación Shiny que permita realizar los principales análisis del paquete sin necesidad de programar.

\textbf{Palabras Clave: ensayos multiambientales, estadística, interfaz gráfica, programación}

%% Por si alguien tiene curiosidad, este "simp'atico" agradecimiento est'a
%% tomado de la "Tesis de Lydia Chalmers" basada en el universo del programa
%% de televisi'on Buffy, la Cazadora de Vampiros.
%% http://www.buffy-cazavampiros.com/Spiketesis/tesis.inicio.htm
