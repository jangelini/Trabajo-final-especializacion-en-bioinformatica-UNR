%% Los cap'itulos inician con \chapter{T'itulo}, estos aparecen numerados y
%% se incluyen en el 'indice general.
%%
%% Recuerda que aqu'i ya puedes escribir acentos como: 'a, 'e, 'i, etc.
%% La letra n con tilde es: 'n.


\chapter*{Abreviaturas}
%\thispagestyle{empty}
\begin{spacing}{1}
\begin{description}
\item{\textbf{ACP}}: análisis de componentes principales.

\item{\textbf{AEC}}: coordenada ambiental promedio (siglas en inglés de \emph{Average environment coordination}).

\item{\textbf{ANOVA}}: análisis de la variancia (siglas en inglés de \emph{analysis of variance}).

\item{\textbf{AMMI}}: modelo de efectos principales aditivos e interacción multiplicativa (siglas en inglés de \emph{Additive Main effects and Multiplicative Interaction}).

\item{\textbf{CCT}}: Centro Científico Teconológico.

\item{\textbf{COI}}: interacción con cambio de rango (siglas en inglés de \emph{crossover interaction}).

\item{\textbf{CONICET}}: Consejo Nacional de Investigaciones Científicas y Técnicas.

\item{\textbf{CRAN}}: \emph{Comprehensive R Archive Network}.

\item{\textbf{DVS}}: descomposición de valores singulares.

\item{\textbf{EM}}: maximización de la esperanza (siglas en inglés de \emph{Expectation Maximization}).

\item{\textbf{EMA}}: ensayos multiambientales.

\item{\textbf{G}}: efecto genotípico.

\item{\textbf{GE}}: genotipo-ambiente (siglas en inglés de \emph{Genotype-Environment}).

\item{\textbf{GGE}}: genotipo más genotipo-ambiente (siglas en inglés de \emph{Genotype plus Genotype-Environment}).

\item{\textbf{IGA}}: interacción genotipo-ambiente.

\item{\textbf{NCOI}}: interacción sin cambio de rango (siglas en inglés de \emph{no crossover interaction}).

\item{\textbf{SREG}}: modelo de regresión por sitio (siglas en inglés de \emph{Site Regression model}).

\item{\textbf{SVP}}: partición de los valores singulares (siglas en inglés de \emph{Singular Value Partition}).

\item{\textbf{ui}}: interfaz del usuario (siglas en inglés de \emph{user interface}).


\end{description}
\end{spacing}