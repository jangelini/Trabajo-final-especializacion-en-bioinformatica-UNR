%% Los cap'itulos inician con \chapter{T'itulo}, estos aparecen numerados y
%% se incluyen en el 'indice general.
%%
%% Recuerda que aqu'i ya puedes escribir acentos como: 'a, 'e, 'i, etc.
%% La letra n con tilde es: 'n.


\chapter*{Abreviaturas y Símbolos}
%\thispagestyle{empty}
\begin{description}
\item{\textbf{ACP}}: análisis de componentes principales.

\item{\textbf{ANOVA}}: análisis de la variancia, del inglés \emph{analysis of variance}.

\item{\textbf{AMMI}}: modelo de efectos principales aditivos e interacción multiplicativa, del inglés \emph{Additive Main effects and Multiplicative Interaction}.

\item{\textbf{COI}}: interacción con cambio de rango, del inglés \emph{crossover interaction}.

\item{\textbf{CRAN}}: \emph{Comprehensive R Archve Network}

\item{\textbf{DVS}}: descomposición de valores singulares

\item{\textbf{EMA}}: ensayos multiambientales.

\item{\textbf{GE}}: Genotipo-ambiente, del inglés \emph{Genotype-Environment}.

\item{\textbf{GGE}}: Genotipo más Genotipo-ambiente, del inglés \emph{Genotype plus Genotype-Environment}.

\item{\textbf{IGA}}: interacción genotipo ambiente.

\item{\textbf{NCOI}}: interacción sin cambio de rango, del inglés \emph{no crossover interaction}.

\item{\textbf{SREG}}: modelo de regresión por sitio, del inglés \emph{Site Regression model}.


\end{description}
